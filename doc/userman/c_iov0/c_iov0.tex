\chapter{Introduction and Overview}
\label{ciov0}

%%%%%%%%%%%%%%%%%%%%%%%%%%%%%%%%%%%%%%%%%%%%%%%%%%%%%%%%%%%%%%%%%%%%%%%%%%%%%%%
%%%%%%%%%%%%%%%%%%%%%%%%%%%%%%%%%%%%%%%%%%%%%%%%%%%%%%%%%%%%%%%%%%%%%%%%%%%%%%%
%%%%%%%%%%%%%%%%%%%%%%%%%%%%%%%%%%%%%%%%%%%%%%%%%%%%%%%%%%%%%%%%%%%%%%%%%%%%%%%
\section{Overview of the Library}
\label{ciov0:sdpr0}

%LaTeX won't substitute into \index{} below, so had to use the expanded
%name.
\index{LibNum@\emph{LibNum}}The library described in this 
document, the \emph{\productbasenamelong{}, Version 
\productversion{}} (identified more compactly as 
\emph{\productbasenameshort{}-\productversion{}}, 
\emph{\productbasenameshort{}}, or 
\emph{\productbasenameultrashort{}}) is a 
\emph{C}/\emph{C++}-callable 
arith\-me\-tic/\-math\-e\-mat\-i\-cal/\-cryp\-to\-graph\-ic 
software library designed for microcontrollers and embedded
systems (however, it is also suitable for personal computers
and servers).

The library includes functionality for: 

\begin{itemize}
\item \emph{Arithmetic:}
      fundamental\footnote{\emph{Fundamental integer} is used here
      to mean an integer type
      specified by the C standard and built into the compiler.} integer,
      large\footnote{\emph{Large integer} is used here to mean an integer
      larger than any fundamental integer type.  Such integers are implemented as
      arrays of fundamental integers.} integer, fixed-point, large fixed-point,
      and float\-ing-point addition, subtraction,
      multiplication, and division.
\item \emph{Standard functions:}
      root extraction, logarithm and exponentiation, trigonometric, and similar
      functions.
\item \emph{Mathematical algorithms:}
      fundamental integer, large integer, fixed-point, large fixed-point,
      floating-point, and vector algorithms.
\item \emph{Control system elements and linear filtering:}
      low/high-pass filtering, scaling, differentiation,
      and integration.
\item \emph{Non-linear filtering:}
      discrete input debouncing, and vertical counters.
\item \emph{Non-cryptographic hashes:}
      checksums, CRC's.
\item \emph{Cryptographic hashes:}
      hashes from the SHA-2 family (SHA-224, SHA-256,
      SHA-384, SHA-512, SHA-512/224, and SHA-512/256).
\item \emph{Cryptography:}
      ciphers and PRNGs.
\item \emph{Searching, Sorting, Insertion, and Deletion:}
      searching for elements in unsorted and sorted tables and arrays,
      sorting tables and arrays, inserting and deleting elements from
      unsorted and sorted tables and arrays.
\item \emph{Bit Manipulation:}
      bit cardinality, searches for set and clear bits, bit manipulation.
\item \emph{Utility:}
      block memory
      operations and array operations.
\end{itemize}

\emph{\productbasenameshort{}} has the following features that make it suitable
for use in embedded systems:

\begin{itemize}
\item The library functions work primarily with fixed-size operands.
\item The library functions are thread-safe, and:
      \begin{itemize}
      \item Can be used in pre-emptive operating systems.
      \item Can be used in multi-core systems.
      \item Can be used in interrupt service routines.
      \end{itemize}
\item The low-level operations of the library are implemented in assembly language
      for efficiency.
\item The library functions use minimal stack space.
\item A templating tool is used to produce the library in several forms:
      \begin{itemize}
      \item A classic library, one function or data structure per compiled module,
            so that the linker extracts the minimum required to link (this form automatically
            gives the minimum FLASH footprint).
      \item Source files with related functions and data structures together (in appearance,
            resembling what a human programmer might produce if FLASH consumption was not
            a concern).
      \end{itemize}
\item The library, as generated for a specific processor/tool combination, has almost no
      preprocessor switches.  (The templating tool produces a customized variant
      of the library for each processor/tool combination.  The elimination of preprocessor
      switches creates code that is easier to understand and does not confound debugging tools.)
\end{itemize}


%%%%%%%%%%%%%%%%%%%%%%%%%%%%%%%%%%%%%%%%%%%%%%%%%%%%%%%%%%%%%%%%%%%%%%%%%%%%%%%
%%%%%%%%%%%%%%%%%%%%%%%%%%%%%%%%%%%%%%%%%%%%%%%%%%%%%%%%%%%%%%%%%%%%%%%%%%%%%%%
%%%%%%%%%%%%%%%%%%%%%%%%%%%%%%%%%%%%%%%%%%%%%%%%%%%%%%%%%%%%%%%%%%%%%%%%%%%%%%%
\section{Acknowledgments}
\label{ciov0:sack0}
 
TBD. 


%%%%%%%%%%%%%%%%%%%%%%%%%%%%%%%%%%%%%%%%%%%%%%%%%%%%%%%%%%%%%%%%%%%%%%%%%%%%%%%
%%%%%%%%%%%%%%%%%%%%%%%%%%%%%%%%%%%%%%%%%%%%%%%%%%%%%%%%%%%%%%%%%%%%%%%%%%%%%%%
%%%%%%%%%%%%%%%%%%%%%%%%%%%%%%%%%%%%%%%%%%%%%%%%%%%%%%%%%%%%%%%%%%%%%%%%%%%%%%%
\section{Feedback, Suggestions, and Collaboration Opportunities}
\label{ciov0:sfbk0}

I welcome all feedback and suggestions about the library and 
documentation, and I welcome all opportunities to collaborate to extend 
the functionality or supported platforms of the library.  

Please feel free to correspond with me at \texttt{dashley@\-gmail.com}.


%%%%%%%%%%%%%%%%%%%%%%%%%%%%%%%%%%%%%%%%%%%%%%%%%%%%%%%%%%%%%%%%%%%%%%%%%%%%%%%
%%%%%%%%%%%%%%%%%%%%%%%%%%%%%%%%%%%%%%%%%%%%%%%%%%%%%%%%%%%%%%%%%%%%%%%%%%%%%%%
%%%%%%%%%%%%%%%%%%%%%%%%%%%%%%%%%%%%%%%%%%%%%%%%%%%%%%%%%%%%%%%%%%%%%%%%%%%%%%%
\section{Licensing and License Interpretation}
\label{ciov0:slip0}

The \emph{\productbasenameshort{}} library and all 
documentation is licensed under
\index{Unlicense@\emph{The Unlicense}}\emph{The Unlicense}.  The 
license text below (\emph{The Unlicense}) is included in the 
\emph{LICENSE} file in the root directory of the project.  

\begin{small}
\begin{verbatim}
This is free and unencumbered software released into the public domain.

Anyone is free to copy, modify, publish, use, compile, sell, or
distribute this software, either in source code form or as a compiled
binary, for any purpose, commercial or non-commercial, and by any
means.

In jurisdictions that recognize copyright laws, the author or authors
of this software dedicate any and all copyright interest in the
software to the public domain. We make this dedication for the benefit
of the public at large and to the detriment of our heirs and
successors. We intend this dedication to be an overt act of
relinquishment in perpetuity of all present and future rights to this
software under copyright law.

THE SOFTWARE IS PROVIDED "AS IS", WITHOUT WARRANTY OF ANY KIND,
EXPRESS OR IMPLIED, INCLUDING BUT NOT LIMITED TO THE WARRANTIES OF
MERCHANTABILITY, FITNESS FOR A PARTICULAR PURPOSE AND NONINFRINGEMENT.
IN NO EVENT SHALL THE AUTHORS BE LIABLE FOR ANY CLAIM, DAMAGES OR
OTHER LIABILITY, WHETHER IN AN ACTION OF CONTRACT, TORT OR OTHERWISE,
ARISING FROM, OUT OF OR IN CONNECTION WITH THE SOFTWARE OR THE USE OR
OTHER DEALINGS IN THE SOFTWARE.

For more information, please refer to <https://unlicense.org>
\end{verbatim}
\end{small}

My interpretation of \emph{The Unlicense} is that it places no restrictions
on a user of the software, other than that they may not litigate against
the authors or curators of the software.  In particular, the software may
be used in any product, including embedded products, without the obligation
to notify the end user or to publish changes.

All documentation and images, in any form (source, intermediate, final, .PDF,
.DVI, etc.), including this document in any form, are also licensed under [the same
terms as] \emph{The Unlicense}. (The documentation is closely tied to the software,
and any errors in the software are likely to be mirrored in the documentation,
and vice-versa.)  With respect to all documentation in any form, including this
document in any form, per the
terms of \emph{The Unlicense}, there is no warranty of any kind, no liability,
no obligation to notify the end user, and
no obligation to publish changes.


%%%%%%%%%%%%%%%%%%%%%%%%%%%%%%%%%%%%%%%%%%%%%%%%%%%%%%%%%%%%%%%%%%%%%%%%%%%%%%%
%%%%%%%%%%%%%%%%%%%%%%%%%%%%%%%%%%%%%%%%%%%%%%%%%%%%%%%%%%%%%%%%%%%%%%%%%%%%%%%
%%%%%%%%%%%%%%%%%%%%%%%%%%%%%%%%%%%%%%%%%%%%%%%%%%%%%%%%%%%%%%%%%%%%%%%%%%%%%%%
\section{Detailed Description of this Document}
\label{ciov0:sotd0}

TBD.


%\emph{\textbf{Part I: General Information}} provides introductory
%and general information about \emph{\productbasenameshort{}}.
%
%\begin{itemize}
%\item \emph{\textbf{Chapter\postchapterwordnonstretchable{}\ref{ciov0}: 
%      Introduction and Overview}} provides general information
%      about \emph{\productbasenameshort{}}.  
%\item \emph{\textbf{Chapter\postchapterwordnonstretchable{}\ref{cldd0}: 
%      Detailed Description of \productbasenameshort{}}}
%      provides a detailed description 
%      of the general features and characteristics
%      of \emph{\productbasenameshort{}}\@.  The specific
%      functions and constant data structures 
%      in \emph{\productbasenameshort{}} are described in
%      Chapters \ref{cuuc0} through \ref{ccip1}.
%\end{itemize}
%
%\emph{\textbf{Part II: Technical Background}} provides mathematical and 
%technical background information useful in understanding the library.  
%
%\begin{itemize}
%\item \emph{\textbf{Chapter\postchapterwordnonstretchable{}\ref{cgct0}: 
%      General Computing}} provides technical background on a variety of general 
%      computing topics.  
%\item \emph{\textbf{Chapter\postchapterwordnonstretchable{}\ref{cbal0}: 
%      Boolean Algebra and Digital Logic}} provides technical background on 
%      Boolean algebra and its application to digital logic, simplifying software 
%      constructs, and vertical counters.  
%\item \emph{\textbf{Chapter\postchapterwordnonstretchable{}\ref{cpta0}: 
%      Polytopes, Automata, and Timed Automata}} provides technical background 
%      related to the design of state machines and stateful systems.  
%\item \emph{\textbf{Chapter\postchapterwordnonstretchable{}\ref{cnth0}: 
%      Number Theory}} provides technical background on number theory.  (Number 
%      theory is the branch of mathematics that deals with positive integers and 
%      their properties, and is the basis of much of modern cryptography.) 
%\item \emph{\textbf{Chapter\postchapterwordnonstretchable{}\ref{crla1}: 
%      Farey Series Approximation}} provides technical background and algorithms 
%      surrounding approximating a real number with a rational number (for 
%      example, 355/113 is the best approximation to $\pi$ with numerator and 
%      denominator not exceeding 1,000).  In rare scenarios, using a rational 
%      approximation to a real number is the most efficient way to perform a 
%      calculation.  Efficiently finding the best rational approximation to a 
%      real number requires\footnote{For example, suppose it is necessary to find 
%      the best rational approximation to $\pi$ with numerator and denominator 
%      not exceeding $2^{256}-1$.  No $O(n)$ algorithm will suffice, as 
%      $2^{256}-1$ is too large.  The actual best approximation is 
%      110,\-859,\-348,\-167,\-216,\-110,\-469,\-450,\-% 
%      301,\-947,\-424,\-979,\-270,\-176,\-705,\-044,\-% 
%      426,\-927,\-758,\-107,\-559,\-970,\-506,\-471,\-% 
%      518,\-580 / 35,\-287,\-626,\-497,\-515,\-783,\-% 
%      106,\-785,\-958,\-721,\-190,\-659,\-891,\-744,\-% 
%      246,\-448,\-011,\-713,\-951,\-598,\-935,\-229,\-% 
%      278,\-291,\-942,\-269---a result that can only be obtained using the better 
%      algorithm presented in this chapter.} results from number theory, which 
%      are presented in this chapter.  
%\item \emph{\textbf{Chapter\postchapterwordnonstretchable{}\ref{ccth0}: 
%      Coding Theory}} provides technical background on error detecting codes, 
%      error correcting codes, and other topics from coding theory.  
%\item \emph{\textbf{Chapter\postchapterwordnonstretchable{}\ref{cnna0}: 
%      Non-Numerical Algorithms}} provides technical background on a variety 
%      non-numerical algorithms that appear in the library, such as algorithms 
%      for searching and sorting.  
%\item \emph{\textbf{Chapter\postchapterwordnonstretchable{}\ref{caal0}: 
%      Integer Arithmetic Algorithms and Implementation}} provides technical 
%      background on integer arithmetic algorithms.  
%\item \emph{\textbf{Chapter\postchapterwordnonstretchable{}\ref{cmal0}: 
%      Integer Mathematical Algorithms and Implementation}} provides technical 
%      background on integer mathematical algorithms.  
%\item \emph{\textbf{Chapter\postchapterwordnonstretchable{}\ref{caal2}: 
%      Fixed-Point Arithmetic Algorithms and Implementation}} provides technical 
%      background on fixed-point arithmetic algorithms.  
%\item \emph{\textbf{Chapter\postchapterwordnonstretchable{}\ref{cmal2}: 
%      Fixed-Point Mathematical Algorithms and Implementation}} provides 
%      technical background on fixed-point mathematical algorithms.  
%\item \emph{\textbf{Chapter\postchapterwordnonstretchable{}\ref{craa0}: 
%      Real and Floating-Point Arithmetic Algorithms and Implementation}} 
%      provides technical background on real and floating-point arithmetic 
%      algorithms.  
%\item \emph{\textbf{Chapter\postchapterwordnonstretchable{}\ref{crma0}: 
%      Real and Floating-Point Mathematical Algorithms and Implementation}} 
%      provides technical background on real and floating-point mathematical 
%      algorithms.  
%\item \emph{\textbf{Chapter\postchapterwordnonstretchable{}\ref{clfc0}: 
%      Linear Filters and Control System Elements}} provides technical background 
%      on linear filters and control system elements.  
%\item \emph{\textbf{Chapter\postchapterwordnonstretchable{}\ref{cnlf0}: 
%      Non-Linear Filters and Debouncing}} provides technical background on 
%      non-linear filters and debouncing.  
%\item \emph{\textbf{Chapter\postchapterwordnonstretchable{}\ref{crng0}: 
%      Random and Pseudo-Random Number Generation}} provides technical background 
%      on the generation of random and pseudo-random numbers.  
%\item \emph{\textbf{Chapter\postchapterwordnonstretchable{}\ref{cnch0}: 
%      Non-Cryptographic Hashes}} provides technical background on 
%      non-cryptographic hashes.  
%\item \emph{\textbf{Chapter\postchapterwordnonstretchable{}\ref{cchs0}: 
%      Cryptographic Hashes}} provides technical background on cryptographic 
%      hashes.  
%\item \emph{\textbf{Chapter\postchapterwordnonstretchable{}\ref{cskc0}: 
%      Symmetric-Key Ciphers and Algorithms}} provides technical background on 
%      symmetric key ciphers and related algorithms.  
%\item \emph{\textbf{Chapter\postchapterwordnonstretchable{}\ref{cakc0}: 
%      Asymmetric-Key Ciphers and Algorithms}} provides technical background on 
%      asymmetric key ciphers and related algorithms.  
%\item \emph{\textbf{Chapter\postchapterwordnonstretchable{}\ref{cmat0}: 
%      Miscellaneous Mathematical and Algorithmic Topics}} provides technical 
%      background on mathematical and algorithmic topics that could not easily be 
%      categorized elsewhere.  
%\end{itemize}
%
%\emph{\textbf{Part II: Library Documentation}} describes
%usage of the library and the actual library functions.
%
%\begin{itemize}
%\item \emph{\textbf{Chapter\postchapterwordnonstretchable{}\ref{cuuc0}: 
%      How to Use \productbasenameshort{}}} explains how to use 
%      \productbasenameshort{} in a project.  
%\item \emph{\textbf{Chapter\postchapterwordnonstretchable{}\ref{cnef0}: 
%      Utility and Miscellaneous Functions}} documents functions that provide 
%      useful non-data-driven functionality.  
%\item \emph{\textbf{Chapter\postchapterwordnonstretchable{}\ref{cbmf0}: 
%      Block Memory Functions}} documents functions that operate on blocks of 
%      memory (setting, copying, shifting, etc.).  
%\item \emph{\textbf{Chapter\postchapterwordnonstretchable{}\ref{csea0}: 
%      Search Functions}} documents functions that perform searches; such as 
%      linear searches and binary searches.
%\item \emph{\textbf{Chapter\postchapterwordnonstretchable{}\ref{csol0}: 
%      Sort Functions}} documents functions that re-order arrays.  
%\item \emph{\textbf{Chapter\postchapterwordnonstretchable{}\ref{cami0}: 
%      Array Manipulation Functions}} documents functions that manipulate arrays.  
%\item \emph{\textbf{Chapter\postchapterwordnonstretchable{}\ref{cbsf0}: 
%      Bit-Mapped Set Functions}} documents functions that operate on sets 
%      represented as arrays of bits.  
%\item \emph{\textbf{Chapter\postchapterwordnonstretchable{}\ref{cvco0}: 
%      Vertical Counter Functions}} documents functions that implement vertical 
%      counters.  
%\item \emph{\textbf{Chapter\postchapterwordnonstretchable{}\ref{cafn0}: 
%      Basic Data Type Integer Utility and Arithmetic Functions}} documents 
%      utility and arithmetic functions that operate on basic (built into the 
%      compiler) character and integer types.  
%\item \emph{\textbf{Chapter\postchapterwordnonstretchable{}\ref{cbaf0}: 
%      Basic Data Type Integer Mathematical Functions}} documents mathematical 
%      functions that operate on basic (built into the compiler) character and 
%      integer types.  
%\item \emph{\textbf{Chapter\postchapterwordnonstretchable{}\ref{cfpa0}: 
%      Basic Data Type Fixed-Point Utility and Arithmetic Functions}} documents 
%      utility and arithmetic functions that operate on fixed-point values stored 
%      in basic (built into the compiler) character and integer types.  
%\item \emph{\textbf{Chapter\postchapterwordnonstretchable{}\ref{cfpa1}: 
%      Basic Data Type Fixed-Point Mathematical Functions}} documents 
%      mathematical functions that operate on fixed-point values stored in basic 
%      (built into the compiler) character and integer types.  
%\item \emph{\textbf{Chapter\postchapterwordnonstretchable{}\ref{caal1}: 
%      Basic Data Type Floating-Point Utility and Arithmetic Functions}} 
%      documents utility and arithmetic functions that operate on basic (built 
%      into the compiler) floating-point types.  
%\item \emph{\textbf{Chapter\postchapterwordnonstretchable{}\ref{cafn1}: 
%      Basic Data Type Floating-Point Mathematical Functions}} documents 
%      mathematical functions that operate on basic (built into the compiler) 
%      floating-point types.  
%\item \emph{\textbf{Chapter\postchapterwordnonstretchable{}\ref{claf0}: 
%      Large Integer Utility and Arithmetic Functions}} documents utility and 
%      arithmetic functions that involve large integers.  
%\item \emph{\textbf{Chapter\postchapterwordnonstretchable{}\ref{claf1}: 
%      Large-Integer Mathematical Functions}} documents mathematical functions 
%      that involve large integers.  
%\item \emph{\textbf{Chapter\postchapterwordnonstretchable{}\ref{cfpa2}: 
%      Large Fixed-Point Utility and Arithmetic Functions}} documents utility and 
%      arithmetic functions that operate on fixed-point values stored as large 
%      integers.  
%\item \emph{\textbf{Chapter\postchapterwordnonstretchable{}\ref{cfpa3}: 
%      Large Fixed-Point Mathematical Functions}} documents mathematical 
%      functions that operate on fixed-point values stored as large integers.  
%\item \emph{\textbf{Chapter\postchapterwordnonstretchable{}\ref{claf2}: 
%      Large Floating-Point Utility and Arithmetic Functions}} documents utility 
%      and arithmetic functions that perform arithmetic on floating-point types 
%      larger than basic floating-point data types.  
%\item \emph{\textbf{Chapter\postchapterwordnonstretchable{}\ref{claf3}: 
%      Large Floating-Point Mathematical Functions}} documents utility and 
%      arithmetic functions that perform arithmetic on floating-point types 
%      larger than basic floating-point data types.  
%\item \emph{\textbf{Chapter\postchapterwordnonstretchable{}\ref{clfi0}: 
%      Linear Filter and Control System Component Functions}} documents functions 
%      that implement classic discrete-time linear filters.  
%\item \emph{\textbf{Chapter\postchapterwordnonstretchable{}\ref{cnfi0}: 
%      Non-Linear Filter Functions}} documents functions that implement 
%      non-linear discrete-time filters.  
%\item \emph{\textbf{Chapter\postchapterwordnonstretchable{}\ref{crng1}: 
%      Pseudo-Random Number Generation Functions}} documents functions that generate
%      pseudo-random numbers by a variety of algorithms.
%\item \emph{\textbf{Chapter\postchapterwordnonstretchable{}\ref{ccrc0}: 
%      Non-Cryptographic Hash Functions}} documents functions that calculate and 
%      manipulate CRCs, checksums, and non-cryptographic hashes.  
%\item \emph{\textbf{Chapter\postchapterwordnonstretchable{}\ref{ccrh0}: 
%      Cryptographic Hash Functions}} documents functions that implement 
%      cryptographic hashes.  
%\item \emph{\textbf{Chapter\postchapterwordnonstretchable{}\ref{ccip0}: 
%      Symmetric Cipher Functions}} documents symmetric cipher functions that 
%      encrypt and decrypt data.  
%\item \emph{\textbf{Chapter\postchapterwordnonstretchable{}\ref{ccip1}: 
%      Asymmetric Cipher Functions}} documents asymmetric cipher functions that 
%      encrypt and decrypt data.  
%\end{itemize}
%
%\emph{\textbf{Part IV: Developer Information}} provides 
%information about building the library from source code and
%extending the library.
%
%\begin{itemize}
%\item \emph{\textbf{Chapter\postchapterwordnonstretchable{}\ref{cbpc0}: 
%      Library Development and Modification Procedures}} 
%      documents how to generate the library source code from
%      the templates, how to modify the templates, how to
%      modify the generator, how to add new processor variants,
%      how to tune the library, how to build the library, and how
%      to test the library.  Design and coding standards are also
%      documented.
%\end{itemize}
%
%\emph{\textbf{Part V: Appendices, Bibliography, and Index}} 
%provides glossaries, references, and an index.  Individuals, 
%products, companies, websites, and Internet newsgroups are 
%cited in the same framework as traditional references in 
%order to provide the reader with more resources to obtain 
%information.  

