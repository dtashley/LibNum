\chapter{Introduction and Overview}
\label{ciov0}

%%%%%%%%%%%%%%%%%%%%%%%%%%%%%%%%%%%%%%%%%%%%%%%%%%%%%%%%%%%%%%%%%%%%%%%%%%%%%%%
%%%%%%%%%%%%%%%%%%%%%%%%%%%%%%%%%%%%%%%%%%%%%%%%%%%%%%%%%%%%%%%%%%%%%%%%%%%%%%%
%%%%%%%%%%%%%%%%%%%%%%%%%%%%%%%%%%%%%%%%%%%%%%%%%%%%%%%%%%%%%%%%%%%%%%%%%%%%%%%
\section{Overview of the Library}
\label{ciov0:sdpr0}

%LaTeX won't substitute into \index{} below, so had to use the expanded
%name.
\index{LibNum@\emph{LibNum}}The library described in this 
document, the \emph{\productbasenamelong{}, Version 
\productversion{}} (identified more compactly as 
\emph{\productbasenameshort{}-\productversion{}}, 
\emph{\productbasenameshort{}}, or 
\emph{\productbasenameultrashort{}}) is a 
\emph{C}/\emph{C++}-callable 
arith\-me\-tic/\-math\-e\-mat\-i\-cal/\-cryp\-to\-graph\-ic 
software library for microcontrollers and embedded systems.  

The library includes functionality for: 

\begin{itemize}
\item \emph{Arithmetic:}
      native, large integer, fixed-point, large fixed-point,
      and float\-ing-point addition, subtraction,
      multiplication, and division.
\item \emph{Mathematical algorithms:}
      native, large integer, fixed-point, large fixed-point,
      floating-point, vector arithmetic, and algebraic 
      and transcendental functions.
\item \emph{Control system elements and linear filtering:}
      low/high-pass filtering, scaling, differentiation,
      and integration.
\item \emph{Non-linear filtering:}
      discrete input debouncing, and vertical counters.
\item \emph{Non-cryptographic hashes:}
      checksums, CRC's.
\item \emph{Cryptographic hashes:}
      hashes from the SHA-2 family (SHA-224, SHA-256,
      SHA-384, SHA-512, SHA-512/224, and SHA-512/256).
\item \emph{Cryptography:}
      ciphers and PRNGs.
\item \emph{Bit-mapped sets:}
      union, intersection, cardinality, and iteration.
\item \emph{Utility:}
      block memory
      operations and array operations.
\end{itemize}

\emph{\productbasenameshort{}} is a traditional
library in that:

\begin{itemize}
\item The source code is arranged as one function or data structure per source file.
\item The source code is compiled as one function or data structure per object file.
\item Object files are combined together into a library.
\item The linker extracts only what it needs to resolve references.  (This generally
      gives minimum FLASH and RAM footprint.)
\end{itemize}

\emph{\productbasenameshort{}} is similar in spirit to 
existing libraries such as \emph{The GNU Multiple Precision 
Arithmetic Library}\footnote{Hereinafter called \emph{The 
GMP} or \emph{GMP}.}, but it has some differences that make 
it more suitable for embedded software.  The features and 
characteristics of the \emph{\productbasenameshort{}} are: 

\begin{itemize}
\item The library is optimized for speed (similar to \emph{The GMP}).
      (Assembly-language for common
      processor variants is provided, with a slower C fallback.  Fixed-size operands
      allow complete loop-unrolling in many cases.)
\item The library seeks to use FLASH memory as economically as possible.  (The design
      of the library allows the linker to extract the minimum required, and the code is
      compacted as much as possible without materially sacrificing speed.)
\item The library seeks to use RAM as economically as possible (but when a speed/RAM
      tradeoff exists, speed is chosen so long as the RAM cost is not unreasonable).
\item Dynamic memory allocation is not used within the library
      (unlike \emph{The GMP}).\footnote{It is still
      possible for the library to operate using dynamically-allocated memory,
      but this memory must be obtained and released by the caller.}
\item All functions in the library operate on operands of a size known by the
      caller, and this data size does not change as a result of any
      library function (unlike \emph{The GMP}).\footnote{All functions
      either accept operands of a fixed
      size characteristic of the function, or accept a measure of size
      as a formal parameter.}
\item The source code for each library variant is created from templates,
      so that the source code contains almost no C preprocessor
      switches.  (The source code is arranged in this way for
      easier debugging,\footnote{Some source-level debuggers are
      confused by complex preprocessor switches.} and to remove uncertainty about what has
      is actually being compiled or assembled.)
\item The library can be repackaged as three source files (a \emph{.h} header file, a \emph{.c}
      source file, and an assembly-language file with a target-specific file extension),
      and used as two software modules within a project rather than as a library.\footnote{The
      library source code is designed to be used in this way because some debugging tools
      are awkward at debugging libraries.}\textsuperscript{,}\footnote{These three files are created automatically
      by the generator program, but it would take typically take substantial manual editing to
      eliminate unused functions and data structures from these files so that the FLASH footprint
      is the same as if a library is used.}
\item The library is made available only as source code.\footnote{The user must
      have the software tools necessary to compile the source code and create a
      library.}
\item The library is thread-safe and core-safe.
\item The library is designed to be used on multi-core processors without
      a cache coherency protocol.\footnote{However, the library achieves this by
      requiring the caller to perform all necessary cache line invalidations
      and flushes.}
\item The library is unit tested to MC/DC coverage, and is 
      expected to contain no software defects.  
\end{itemize}


%%%%%%%%%%%%%%%%%%%%%%%%%%%%%%%%%%%%%%%%%%%%%%%%%%%%%%%%%%%%%%%%%%%%%%%%%%%%%%%
%%%%%%%%%%%%%%%%%%%%%%%%%%%%%%%%%%%%%%%%%%%%%%%%%%%%%%%%%%%%%%%%%%%%%%%%%%%%%%%
%%%%%%%%%%%%%%%%%%%%%%%%%%%%%%%%%%%%%%%%%%%%%%%%%%%%%%%%%%%%%%%%%%%%%%%%%%%%%%%
\section{Acknowledgments}
\label{ciov0:sack0}
 
TBD. 


%%%%%%%%%%%%%%%%%%%%%%%%%%%%%%%%%%%%%%%%%%%%%%%%%%%%%%%%%%%%%%%%%%%%%%%%%%%%%%%
%%%%%%%%%%%%%%%%%%%%%%%%%%%%%%%%%%%%%%%%%%%%%%%%%%%%%%%%%%%%%%%%%%%%%%%%%%%%%%%
%%%%%%%%%%%%%%%%%%%%%%%%%%%%%%%%%%%%%%%%%%%%%%%%%%%%%%%%%%%%%%%%%%%%%%%%%%%%%%%
\section{Feedback, Suggestions, and Collaboration Opportunities}
\label{ciov0:sfbk0}

I welcome all feedback and suggestions about the library and 
documentation, and I welcome all opportunities to collaborate to extend 
the functionality or supported platforms of the library.  

Please feel free to correspond with me at \texttt{dashley@\-gmail.com}.


%%%%%%%%%%%%%%%%%%%%%%%%%%%%%%%%%%%%%%%%%%%%%%%%%%%%%%%%%%%%%%%%%%%%%%%%%%%%%%%
%%%%%%%%%%%%%%%%%%%%%%%%%%%%%%%%%%%%%%%%%%%%%%%%%%%%%%%%%%%%%%%%%%%%%%%%%%%%%%%
%%%%%%%%%%%%%%%%%%%%%%%%%%%%%%%%%%%%%%%%%%%%%%%%%%%%%%%%%%%%%%%%%%%%%%%%%%%%%%%
\section{Nomenclature}
\label{ciov0:snom0}

This section defines key terms near the start of the 
document so that the remainder of the document is easier 
reading.  


%%%%%%%%%%%%%%%%%%%%%%%%%%%%%%%%%%%%%%%%%%%%%%%%%%%%%%%%%%%%%%%%%%%%%%%%%%%%%%%
%%%%%%%%%%%%%%%%%%%%%%%%%%%%%%%%%%%%%%%%%%%%%%%%%%%%%%%%%%%%%%%%%%%%%%%%%%%%%%%
%%%%%%%%%%%%%%%%%%%%%%%%%%%%%%%%%%%%%%%%%%%%%%%%%%%%%%%%%%%%%%%%%%%%%%%%%%%%%%%
\subsection{Processor Nomenclature}
\label{ciov0:snom0:sptr0}

A \index{processor family}\emph{processor family} is a 
family of processors, which may not all have the same 
features or be compatible at the machine instruction 
level.\footnote{For example, the \emph{PowerPC} or 
\emph{Power Architecture} family of processors includes 
variants that support the traditional 32-bit instruction 
set, variants that support the Book VLE instruction set, and 
variants that support both.} Examples of processor families 
include ``8086'' or ``PPC''.  

A \index{processor variant}\emph{processor variant} is a set 
of processor models that can be treated identically by 
\emph{\productbasenameshort{}} (the processors can use the 
same machine code and same variant of the 
\emph{\productbasenameshort{}} library).  Because 
\emph{\productbasenameshort{}} uses only a subset of 
processor functionality, processors that differ in some 
details that don't affect \emph{\productbasenameshort{}} are 
considered to be the same processor variant.  

An \index{instruction set extension}\emph{instruction set 
extension} is a set of instructions supported by a processor 
model that isn't supported by all models within the 
processor variant.  An example of an instruction set 
extension is Intel's SHA extensions.  Instruction set 
extensions are handled within \emph{\productbasenameshort{}} 
by using preprocessor switches in the \emph{LibNum.h} file 
to enable or disable the use of the extensions.  The default
for all preprocessor switches is to assume that the 
instruction set extension is not present in the processor 
model.  The end user of the library may modify the default 
values of the switches to enable the use of instruction set 
extensions.

A \index{multi-core processor}\emph{multi-core processor} is 
a single integrated circuit that contains more than one 
processor (or \emph{core}).  

In desktop computing, the cores of a multi-core processor 
are most typically identical in functionality and speed, so 
that a desktop multi-core processor is typically an 
\index{symmetric multiprocessor}\emph{SMP}, or 
\index{symmetric multiprocessor}\emph{symmetric 
multiprocessor}.  

In microcontroller multi-core processors, it is common to 
have cores that accept the same instruction set (and so can 
run the same machine instructions from the same FLASH 
memory), but which differ in speed or architectural 
features.  Although strictly speaking such systems are 
\emph{ASMP} (\index{asymmetric 
multiprocessor}\emph{asymmetric multiprocessor}) systems, 
all cores can usually run the same variant of
\emph{\productbasenameshort{}}, but possibly at different 
rates.  

A processor may execute instructions out of order 
(\index{out-of-order execution}\emph{out-of-order 
execution}, or \index{out-of-order execution}\emph{OOE})\@.  
It is necessary to differentiate between when a machine 
instruction is fetched by the processor for non-speculatve 
execution (\index{instruction fetch}\emph{instruction 
fetch}), and when the instruction has been fully executed 
(\index{instruction retirement}\emph{instruction 
retirement}).  If a processor has more than one instruction 
that is fetched but not retired, it may perform the memory 
operations necessary to execute those 
instructions in any order, so long as the processor 
determines that the end result will be the same as if the 
instructions had been executed one-at-a-time.  This is by 
design not a problem for single-threaded programs, but may 
cause problems for programs that interact with other 
cores.\footnote{This won't normally cause problems for 
multi-threaded programs on a \emph{single} core, because
threads can only be switched as the result of an interrupt, 
and the processor normally handles that case correctly 
without any special actions by the programmer.} 

A \index{memory barrier}\emph{memory barrier instruction} 
(or \emph{memory barrier}) can be used to cause the 
processor to order memory (or I/O) accesses in the same 
order as instruction fetch.  Generally, a memory barrier 
instruction guarantees that all memory accesses necessary to 
execute instructions before the barrier will be completed 
before any memory accesses necessary to execute instructions 
after the barrier.  Processors typically provide several 
memory barrier instructions, differing in which I/O 
subsystems they affect and in the strength of the guarantees 
provided.  

Memory barrier instructions may be necessary in coordinating 
between cores, to ensure that other cores sense memory 
changes in a desired order.  Memory barrier instructions may 
also be necessary in certain types of I/O.  

To improve performance, cores may contain 
\index{cache}\emph{cache}, which is core-local memory 
mirroring the contents of RAM\footnote{FLASH may also be 
cached, and a core may have a separate instruction and data 
cache.  However, the cache coherency problems that apply to 
a mathematical/cryptographic library occur only in the data 
cache and only in the caching of RAM.} on read operations, 
and buffering write operations.  Cache is divided into 
\index{cache!line}\emph{cache lines}, which are the minimum 
unit of data transfer between cache and RAM.  

Since each core may have its own cache and each core 
accesses RAM independently, there is the opportunity for a 
core's cache contents to become inconsistent (or incoherent) 
with the contents of the cache of other cores (in other 
words, two cores may sense the same RAM location as having 
different contents).  The issue of whether caches may have 
inconsistent information about RAM contents is called 
\index{cache!coherency}\emph{cache coherency}.  

Multiprocessors deal with cache coherency in two ways:

\begin{itemize}
\item \emph{Hardware:} special hardware features exist to
      automatically keep caches consistent, and no
      special software actions are required.  
\item \emph{Software:} the problem is not addressed in
      hardware, so software must use special
      steps\footnote{Usually, some combination of 
      memory barrier instructions, cache invalidation
      instructions, and cache flush instructions.} to
      coordinate between cores using RAM.  
\end{itemize}

At almost any time, a core may receive an 
\index{interrupt}\emph{interrupt}, which causes the core to 
immediately begin executing a subroutine associated with the 
source of the interrupt.  This subroutine is called an 
\index{ISR (interrupt service routine)}\emph{interrupt 
service routine}, or \emph{ISR}\@.  It is 
sometimes\footnote{It is not \emph{always} possible.  
Sophisticated processors have features to prevent 
unprivileged code from disabling or enabling interrupts.} 
possible to prevent interrupts from occurring during a 
particular sequence of machine instructions.  


%%%%%%%%%%%%%%%%%%%%%%%%%%%%%%%%%%%%%%%%%%%%%%%%%%%%%%%%%%%%%%%%%%%%%%%%%%%%%%%
%%%%%%%%%%%%%%%%%%%%%%%%%%%%%%%%%%%%%%%%%%%%%%%%%%%%%%%%%%%%%%%%%%%%%%%%%%%%%%%
%%%%%%%%%%%%%%%%%%%%%%%%%%%%%%%%%%%%%%%%%%%%%%%%%%%%%%%%%%%%%%%%%%%%%%%%%%%%%%%
\subsection{Software and Development Tool Nomenclature}
\label{ciov0:snom0:sdvt0}

\index{client software}\emph{Client software} is the 
software which calls the functions in 
\emph{\productbasenameshort{}}.  

Each \emph{C}/\emph{C++} compiler has \index{basic data 
type}\emph{basic data types} (those built into the 
language).  From \emph{C}/\emph{C++}, examples of basic data 
types include \emph{char}, \emph{int}, and \emph{float}.  
Often, compiler options are available to change the sizes of 
some basic data types (allowing an \emph{int} to be either 
16 or 32 bits is common, for example).  Those compiler 
options affect the compatibility of a numeric library in 
binary form, because library functions make data size 
assumptions.  A numeric library in binary form can't be used 
with client software unless the sizes of basic data types 
are known.  Hence, compiler options may affect the 
definition of \emph{library variant} (described below).  
Certain other compiler options (those that affect the use of 
floating-point hardware, for example) may also limit binary 
library compatibility.  

\emph{C}/\emph{C++} compilers operate under rules about how 
one function calls another function (the organization of the 
stack and stack frames, how parameters are passed, which 
processor registers a function is free to modify and which 
must be saved and restored, etc.).  These rules are 
typically specified in a document called an 
\index{application binary interface}\emph{application binary 
interface specification} or \emph{ABI spec.} or 
\emph{ABI}\@.  The ABI naturally affects compatibility of a 
binary library; but also may affect the library source code 
due to the presence of assembly-language.  

A \index{library variant}\emph{library variant} is a binary 
version of \emph{\productbasenameshort{}} to support a 
specific combination of processor variant, compiler, 
compiler options, and ABI\@.  A single processor variant may 
have multiple library variants (due to compiler, compiler 
options, and the ABI).  

A \emph{C}/\emph{C++} compiler may reorder the machine 
instructions corresponding to \emph{C}/\emph{C++}-language 
statements so that the machine instructions appear in a 
different order than the corresponding \emph{C}/\emph{C++} 
statements.  Although this is not a problem for 
single-threaded programs, it may introduce issues for 
multi-threaded programs or for coordination between cores.  
To prevent reordering of machine instructions, a 
\index{compiler barrier}\emph{compiler barrier} may be used.  
A compiler barrier is a special directive provided to the 
compiler to ensure that all \emph{C}/\emph{C++} statements 
before the barrier have their machine instructions occur 
before any of the \emph{C}/\emph{C++} statements after the 
barrier.\footnote{Note that even with use of compiler 
barriers, out-of-order execution 
(\S{}\ref{ciov0:snom0:sptr0}) may still create logical 
problems in coordinating between cores.} 


%%%%%%%%%%%%%%%%%%%%%%%%%%%%%%%%%%%%%%%%%%%%%%%%%%%%%%%%%%%%%%%%%%%%%%%%%%%%%%%
%%%%%%%%%%%%%%%%%%%%%%%%%%%%%%%%%%%%%%%%%%%%%%%%%%%%%%%%%%%%%%%%%%%%%%%%%%%%%%%
%%%%%%%%%%%%%%%%%%%%%%%%%%%%%%%%%%%%%%%%%%%%%%%%%%%%%%%%%%%%%%%%%%%%%%%%%%%%%%%
\subsection{Computer Arithmetic Nomenclature}
\label{ciov0:snom0:scan0}

The set of integers is denoted 
\index{Z@$\vworkintset$}$\vworkintset{}$\@.  The set of 
non-negative integers is denoted 
\index{Z+@$\vworkintsetnonneg$}$\vworkintsetnonneg{}$, and 
the set of positive integers is denoted 
\index{N@$\vworkintsetpos$}$\vworkintsetpos{}$.  

\begin{eqnarray}
\label{eq:ciov0:snom0:scan0:01}
\vworkintset{}   \;    & = & \{ \ldots , -2, -1, 0, 1, 2, \ldots \} \\
\label{eq:ciov0:snom0:scan0:02}
\vworkintsetnonneg{} \!\!\!\!\; & = & \{ 0, 1, 2, \ldots \}         \\
\label{eq:ciov0:snom0:scan0:03}
\vworkintsetpos{}  \;  & = & \{ 1, 2, 3, \ldots \}
\end{eqnarray}

Similarly the set of real numbers is denoted 
\index{R@$\vworkrealset$}$\vworkrealset$, and the set of 
non-negative real numbers is denoted 
\index{R+@$\vworkrealsetnonneg$}$\vworkrealsetnonneg$.  

A \index{native integer}\emph{native integer} is an integer 
that corresponds to a \emph{C}/\emph{C++} native data type.  
These are usually, but not always, integers that the 
processor can manipulate with single machine instructions.  

A \index{large integer}\emph{large integer} is an integer 
that is larger than a \emph{C}/\emph{C++} native data type.  
(In \emph{\productbasenameshort{}}, large integers are 
usually represented as arrays of native integers.  Each such 
native integer is called a \index{limb}\emph{limb}\index{GMP 
Library@\emph{GMP Library}}\footnote{This nomenclature comes 
from \emph{The GMP Library} 
\cite{bibref:w:gmplibhomepage}.}, and the size of each limb, 
called the \index{limb size}\emph{limb size}, is chosen 
based on what integer sizes the processor instruction set 
accommodates.) 

A \index{fixed-point number}\emph{fixed-point number} is an 
integer where each count of the integer corresponds to 
$1/k$, $k \in \vworkintsetpos$ in the represented value.  
$k$ is typically chosen to be a power of 2 so that the radix 
point of the fixed-point number appears between two bits.  

A \emph{native fixed-point number} is a fixed-point number 
implemented using a native integer.  A \emph{large 
fixed-point number} is a fixed-point number implemented 
using a large integer.  

A \index{floating-point number}\emph{floating-point number} 
is a number containing a sign, mantissa, and exponent.  
Floating-point numbers are designed to be an approximate 
representation of $\vworkrealset$.  They have a larger range 
than fixed-point numbers, but precision that depends on the 
value being represented.  

A \emph{native floating-point number} is a floating-point 
number implemented using a native floating-point data type.  
A \emph{large floating-point number} is a floating-point 
number that is larger (in terms of memory storage 
requirements) than a native floating-point number.  


%%%%%%%%%%%%%%%%%%%%%%%%%%%%%%%%%%%%%%%%%%%%%%%%%%%%%%%%%%%%%%%%%%%%%%%%%%%%%%%
%%%%%%%%%%%%%%%%%%%%%%%%%%%%%%%%%%%%%%%%%%%%%%%%%%%%%%%%%%%%%%%%%%%%%%%%%%%%%%%
%%%%%%%%%%%%%%%%%%%%%%%%%%%%%%%%%%%%%%%%%%%%%%%%%%%%%%%%%%%%%%%%%%%%%%%%%%%%%%%
\section{Licensing and License Interpretation}
\label{ciov0:slip0}

The \emph{\productbasenameshort{}} library and all 
documentation is licensed under \emph{The MIT License}.  The 
license below (\emph{The MIT License}) is included in the 
\emph{LICENSE} file in the root directory of the project.  

%Note:  The verbatim text below was formatted in SlickEdit
%       by setting the margin to 72 and selecting each 
%       paragraph then invoking "Format Selection".
\begin{small}
\begin{verbatim}
Copyright 2021 David T. Ashley (dashley@gmail.com)

Permission is hereby granted, free of charge, to any person obtaining a 
copy of this software and associated documentation files (the 
``Software''), to deal in the Software without restriction, including 
without limitation the rights to use, copy, modify, merge, publish, 
distribute, sublicense, and/or sell copies of the Software, and to 
permit persons to whom the Software is furnished to do so, subject to 
the following conditions: 

The above copyright notice and this permission notice shall be included 
in all copies or substantial portions of the Software.  

THE SOFTWARE IS PROVIDED ``AS IS'', WITHOUT WARRANTY OF ANY KIND, 
EXPRESS OR IMPLIED, INCLUDING BUT NOT LIMITED TO THE WARRANTIES OF 
MERCHANTABILITY, FITNESS FOR A PARTICULAR PURPOSE AND NONINFRINGEMENT.  
IN NO EVENT SHALL THE AUTHORS OR COPYRIGHT HOLDERS BE LIABLE FOR ANY 
CLAIM, DAMAGES OR OTHER LIABILITY, WHETHER IN AN ACTION OF CONTRACT, 
TORT OR OTHERWISE, ARISING FROM, OUT OF OR IN CONNECTION WITH THE 
SOFTWARE OR THE USE OR OTHER DEALINGS IN THE SOFTWARE.  
\end{verbatim}
\end{small}

The general consensus in the open-source community about 
\emph{The MIT License} (and my interpretation of the 
license) is:

\begin{itemize}
\item ``Software'' is defined in the first paragraph to 
      be restricted to the original materials made 
      available under the license.  ``Software'' does not 
      include any binary derivatives created by the licensee 
      as a result of compiling or assembling the software, 
      or created by the licensee as a result of integrating 
      these compiled or assembled derivatives into a 
      product, as the binary derivatives are not made 
      available under the license.
\item The license does not require that modified versions
      of the ``Software'' be made public.
\item The license does not require that any copyright, 
      license, or permission information or notices be 
      retained in compiled libraries or in any binary 
      derivatives of the ``Software'' created by the 
      licensee.
\item The license does not require that the user of any 
      program or product in which compiled or assembled 
      derivatives of the ``Software'' are used be notified 
      of the presence of derivatives of the ``Software'' in 
      the product; or be notified of any copyright, license, 
      or permission information involving the use of the 
      ``Software''.
\end{itemize}


%%%%%%%%%%%%%%%%%%%%%%%%%%%%%%%%%%%%%%%%%%%%%%%%%%%%%%%%%%%%%%%%%%%%%%%%%%%%%%%
%%%%%%%%%%%%%%%%%%%%%%%%%%%%%%%%%%%%%%%%%%%%%%%%%%%%%%%%%%%%%%%%%%%%%%%%%%%%%%%
%%%%%%%%%%%%%%%%%%%%%%%%%%%%%%%%%%%%%%%%%%%%%%%%%%%%%%%%%%%%%%%%%%%%%%%%%%%%%%%
\section{Detailed Description of this Document}
\label{ciov0:sotd0}

\emph{\textbf{Part I: General Information}} provides introductory
and general information about \emph{\productbasenameshort{}}.

\begin{itemize}
\item \emph{\textbf{Chapter\postchapterwordnonstretchable{}\ref{ciov0}: 
      Introduction and Overview}} provides general information
      about \emph{\productbasenameshort{}}.  
\item \emph{\textbf{Chapter\postchapterwordnonstretchable{}\ref{cldd0}: 
      Detailed Description of \productbasenameshort{}}}
      provides a detailed description 
      of the general features and characteristics
      of \emph{\productbasenameshort{}}\@.  The specific
      functions and constant data structures 
      in \emph{\productbasenameshort{}} are described in
      Chapters \ref{cuuc0} through \ref{ccip1}.
\end{itemize}

\emph{\textbf{Part II: Technical Background}} provides mathematical and 
technical background information useful in understanding the library.  

\begin{itemize}
\item \emph{\textbf{Chapter\postchapterwordnonstretchable{}\ref{cgct0}: 
      General Computing}} provides technical background on a variety of general 
      computing topics.  
\item \emph{\textbf{Chapter\postchapterwordnonstretchable{}\ref{cbal0}: 
      Boolean Algebra and Digital Logic}} provides technical background on 
      Boolean algebra and its application to digital logic, simplifying software 
      constructs, and vertical counters.  
\item \emph{\textbf{Chapter\postchapterwordnonstretchable{}\ref{cpta0}: 
      Polytopes, Automata, and Timed Automata}} provides technical background 
      related to the design of state machines and stateful systems.  
\item \emph{\textbf{Chapter\postchapterwordnonstretchable{}\ref{cnth0}: 
      Number Theory}} provides technical background on number theory.  (Number 
      theory is the branch of mathematics that deals with positive integers and 
      their properties, and is the basis of much of modern cryptography.) 
\item \emph{\textbf{Chapter\postchapterwordnonstretchable{}\ref{crla1}: 
      Farey Series Approximation}} provides technical background and algorithms 
      surrounding approximating a real number with a rational number (for 
      example, 355/113 is the best approximation to $\pi$ with numerator and 
      denominator not exceeding 1,000).  In rare scenarios, using a rational 
      approximation to a real number is the most efficient way to perform a 
      calculation.  Efficiently finding the best rational approximation to a 
      real number requires\footnote{For example, suppose it is necessary to find 
      the best rational approximation to $\pi$ with numerator and denominator 
      not exceeding $2^{256}-1$.  No $O(n)$ algorithm will suffice, as 
      $2^{256}-1$ is too large.  The actual best approximation is 
      110,\-859,\-348,\-167,\-216,\-110,\-469,\-450,\-% 
      301,\-947,\-424,\-979,\-270,\-176,\-705,\-044,\-% 
      426,\-927,\-758,\-107,\-559,\-970,\-506,\-471,\-% 
      518,\-580 / 35,\-287,\-626,\-497,\-515,\-783,\-% 
      106,\-785,\-958,\-721,\-190,\-659,\-891,\-744,\-% 
      246,\-448,\-011,\-713,\-951,\-598,\-935,\-229,\-% 
      278,\-291,\-942,\-269---a result that can only be obtained using the better 
      algorithm presented in this chapter.} results from number theory, which 
      are presented in this chapter.  
\item \emph{\textbf{Chapter\postchapterwordnonstretchable{}\ref{ccth0}: 
      Coding Theory}} provides technical background on error detecting codes, 
      error correcting codes, and other topics from coding theory.  
\item \emph{\textbf{Chapter\postchapterwordnonstretchable{}\ref{cnna0}: 
      Non-Numerical Algorithms}} provides technical background on a variety 
      non-numerical algorithms that appear in the library, such as algorithms 
      for searching and sorting.  
\item \emph{\textbf{Chapter\postchapterwordnonstretchable{}\ref{caal0}: 
      Integer Arithmetic Algorithms and Implementation}} provides technical 
      background on integer arithmetic algorithms.  
\item \emph{\textbf{Chapter\postchapterwordnonstretchable{}\ref{cmal0}: 
      Integer Mathematical Algorithms and Implementation}} provides technical 
      background on integer mathematical algorithms.  
\item \emph{\textbf{Chapter\postchapterwordnonstretchable{}\ref{caal2}: 
      Fixed-Point Arithmetic Algorithms and Implementation}} provides technical 
      background on fixed-point arithmetic algorithms.  
\item \emph{\textbf{Chapter\postchapterwordnonstretchable{}\ref{cmal2}: 
      Fixed-Point Mathematical Algorithms and Implementation}} provides 
      technical background on fixed-point mathematical algorithms.  
\item \emph{\textbf{Chapter\postchapterwordnonstretchable{}\ref{craa0}: 
      Real and Floating-Point Arithmetic Algorithms and Implementation}} 
      provides technical background on real and floating-point arithmetic 
      algorithms.  
\item \emph{\textbf{Chapter\postchapterwordnonstretchable{}\ref{crma0}: 
      Real and Floating-Point Mathematical Algorithms and Implementation}} 
      provides technical background on real and floating-point mathematical 
      algorithms.  
\item \emph{\textbf{Chapter\postchapterwordnonstretchable{}\ref{clfc0}: 
      Linear Filters and Control System Elements}} provides technical background 
      on linear filters and control system elements.  
\item \emph{\textbf{Chapter\postchapterwordnonstretchable{}\ref{cnlf0}: 
      Non-Linear Filters and Debouncing}} provides technical background on 
      non-linear filters and debouncing.  
\item \emph{\textbf{Chapter\postchapterwordnonstretchable{}\ref{crng0}: 
      Random and Pseudo-Random Number Generation}} provides technical background 
      on the generation of random and pseudo-random numbers.  
\item \emph{\textbf{Chapter\postchapterwordnonstretchable{}\ref{cnch0}: 
      Non-Cryptographic Hashes}} provides technical background on 
      non-cryptographic hashes.  
\item \emph{\textbf{Chapter\postchapterwordnonstretchable{}\ref{cchs0}: 
      Cryptographic Hashes}} provides technical background on cryptographic 
      hashes.  
\item \emph{\textbf{Chapter\postchapterwordnonstretchable{}\ref{cskc0}: 
      Symmetric-Key Ciphers and Algorithms}} provides technical background on 
      symmetric key ciphers and related algorithms.  
\item \emph{\textbf{Chapter\postchapterwordnonstretchable{}\ref{cakc0}: 
      Asymmetric-Key Ciphers and Algorithms}} provides technical background on 
      asymmetric key ciphers and related algorithms.  
\item \emph{\textbf{Chapter\postchapterwordnonstretchable{}\ref{cmat0}: 
      Miscellaneous Mathematical and Algorithmic Topics}} provides technical 
      background on mathematical and algorithmic topics that could not easily be 
      categorized elsewhere.  
\end{itemize}

\emph{\textbf{Part II: Library Documentation}} describes
usage of the library and the actual library functions.

\begin{itemize}
\item \emph{\textbf{Chapter\postchapterwordnonstretchable{}\ref{cuuc0}: 
      How to Use \productbasenameshort{}}} explains how to use 
      \productbasenameshort{} in a project.  
\item \emph{\textbf{Chapter\postchapterwordnonstretchable{}\ref{cnef0}: 
      Utility and Miscellaneous Functions}} documents functions that provide 
      useful non-data-driven functionality.  
\item \emph{\textbf{Chapter\postchapterwordnonstretchable{}\ref{cbmf0}: 
      Block Memory Functions}} documents functions that operate on blocks of 
      memory (setting, copying, shifting, etc.).  
\item \emph{\textbf{Chapter\postchapterwordnonstretchable{}\ref{csea0}: 
      Search Functions}} documents functions that perform searches; such as 
      linear searches and binary searches.
\item \emph{\textbf{Chapter\postchapterwordnonstretchable{}\ref{csol0}: 
      Sort Functions}} documents functions that re-order arrays.  
\item \emph{\textbf{Chapter\postchapterwordnonstretchable{}\ref{cami0}: 
      Array Manipulation Functions}} documents functions that manipulate arrays.  
\item \emph{\textbf{Chapter\postchapterwordnonstretchable{}\ref{cbsf0}: 
      Bit-Mapped Set Functions}} documents functions that operate on sets 
      represented as arrays of bits.  
\item \emph{\textbf{Chapter\postchapterwordnonstretchable{}\ref{cvco0}: 
      Vertical Counter Functions}} documents functions that implement vertical 
      counters.  
\item \emph{\textbf{Chapter\postchapterwordnonstretchable{}\ref{cafn0}: 
      Basic Data Type Integer Utility and Arithmetic Functions}} documents 
      utility and arithmetic functions that operate on basic (built into the 
      compiler) character and integer types.  
\item \emph{\textbf{Chapter\postchapterwordnonstretchable{}\ref{cbaf0}: 
      Basic Data Type Integer Mathematical Functions}} documents mathematical 
      functions that operate on basic (built into the compiler) character and 
      integer types.  
\item \emph{\textbf{Chapter\postchapterwordnonstretchable{}\ref{cfpa0}: 
      Basic Data Type Fixed-Point Utility and Arithmetic Functions}} documents 
      utility and arithmetic functions that operate on fixed-point values stored 
      in basic (built into the compiler) character and integer types.  
\item \emph{\textbf{Chapter\postchapterwordnonstretchable{}\ref{cfpa1}: 
      Basic Data Type Fixed-Point Mathematical Functions}} documents 
      mathematical functions that operate on fixed-point values stored in basic 
      (built into the compiler) character and integer types.  
\item \emph{\textbf{Chapter\postchapterwordnonstretchable{}\ref{caal1}: 
      Basic Data Type Floating-Point Utility and Arithmetic Functions}} 
      documents utility and arithmetic functions that operate on basic (built 
      into the compiler) floating-point types.  
\item \emph{\textbf{Chapter\postchapterwordnonstretchable{}\ref{cafn1}: 
      Basic Data Type Floating-Point Mathematical Functions}} documents 
      mathematical functions that operate on basic (built into the compiler) 
      floating-point types.  
\item \emph{\textbf{Chapter\postchapterwordnonstretchable{}\ref{claf0}: 
      Large Integer Utility and Arithmetic Functions}} documents utility and 
      arithmetic functions that involve large integers.  
\item \emph{\textbf{Chapter\postchapterwordnonstretchable{}\ref{claf1}: 
      Large-Integer Mathematical Functions}} documents mathematical functions 
      that involve large integers.  
\item \emph{\textbf{Chapter\postchapterwordnonstretchable{}\ref{cfpa2}: 
      Large Fixed-Point Utility and Arithmetic Functions}} documents utility and 
      arithmetic functions that operate on fixed-point values stored as large 
      integers.  
\item \emph{\textbf{Chapter\postchapterwordnonstretchable{}\ref{cfpa3}: 
      Large Fixed-Point Mathematical Functions}} documents mathematical 
      functions that operate on fixed-point values stored as large integers.  
\item \emph{\textbf{Chapter\postchapterwordnonstretchable{}\ref{claf2}: 
      Large Floating-Point Utility and Arithmetic Functions}} documents utility 
      and arithmetic functions that perform arithmetic on floating-point types 
      larger than basic floating-point data types.  
\item \emph{\textbf{Chapter\postchapterwordnonstretchable{}\ref{claf3}: 
      Large Floating-Point Mathematical Functions}} documents utility and 
      arithmetic functions that perform arithmetic on floating-point types 
      larger than basic floating-point data types.  
\item \emph{\textbf{Chapter\postchapterwordnonstretchable{}\ref{clfi0}: 
      Linear Filter and Control System Component Functions}} documents functions 
      that implement classic discrete-time linear filters.  
\item \emph{\textbf{Chapter\postchapterwordnonstretchable{}\ref{cnfi0}: 
      Non-Linear Filter Functions}} documents functions that implement 
      non-linear discrete-time filters.  
\item \emph{\textbf{Chapter\postchapterwordnonstretchable{}\ref{crng1}: 
      Pseudo-Random Number Generation Functions}} documents functions that generate
      pseudo-random numbers by a variety of algorithms.
\item \emph{\textbf{Chapter\postchapterwordnonstretchable{}\ref{ccrc0}: 
      Non-Cryptographic Hash Functions}} documents functions that calculate and 
      manipulate CRCs, checksums, and non-cryptographic hashes.  
\item \emph{\textbf{Chapter\postchapterwordnonstretchable{}\ref{ccrh0}: 
      Cryptographic Hash Functions}} documents functions that implement 
      cryptographic hashes.  
\item \emph{\textbf{Chapter\postchapterwordnonstretchable{}\ref{ccip0}: 
      Symmetric Cipher Functions}} documents symmetric cipher functions that 
      encrypt and decrypt data.  
\item \emph{\textbf{Chapter\postchapterwordnonstretchable{}\ref{ccip1}: 
      Asymmetric Cipher Functions}} documents asymmetric cipher functions that 
      encrypt and decrypt data.  
\end{itemize}

\emph{\textbf{Part IV: Developer Information}} provides 
information about building the library from source code and
extending the library.

\begin{itemize}
\item \emph{\textbf{Chapter\postchapterwordnonstretchable{}\ref{cbpc0}: 
      Library Development and Modification Procedures}} 
      documents how to generate the library source code from
      the templates, how to modify the templates, how to
      modify the generator, how to add new processor variants,
      how to tune the library, how to build the library, and how
      to test the library.  Design and coding standards are also
      documented.
\end{itemize}

\emph{\textbf{Part V: Appendices, Bibliography, and Index}} 
provides glossaries, references, and an index.  Individuals, 
products, companies, websites, and Internet newsgroups are 
cited in the same framework as traditional references in 
order to provide the reader with more resources to obtain 
information.  

