\chapter{Detailed Description of \emph{\productbasenameshort{}}}
\label{cldd0}

This chapter describes in detail the general features and characteristics
of \emph{\productbasenameshort{}}\@.  Additionally, Chapters \ref{cuuc0}
through \ref{ccip1} document the individual functions and constant data
structures in the library, and Chapter \ref{cbpc0} documents how to build
and modify the library.


%%%%%%%%%%%%%%%%%%%%%%%%%%%%%%%%%%%%%%%%%%%%%%%%%%%%%%%%%%%%%%%%%%%%%%%%%%%%%%%
%%%%%%%%%%%%%%%%%%%%%%%%%%%%%%%%%%%%%%%%%%%%%%%%%%%%%%%%%%%%%%%%%%%%%%%%%%%%%%%
%%%%%%%%%%%%%%%%%%%%%%%%%%%%%%%%%%%%%%%%%%%%%%%%%%%%%%%%%%%%%%%%%%%%%%%%%%%%%%%
\section{Library Variants}
\label{cldd0:sscc0}

Table \ref{tbl:cldd0:sscc0:01} (p.  
\pageref{tbl:cldd0:sscc0:01}) lists the available library 
variants.  A library variant (\S{}\ref{ciov0:snom0:sdvt0}, 
p.  \pageref{ciov0:snom0:sdvt0}) corresponds to a 
combination of processor variant, compiler, compiler 
options, and ABI; so Table \ref{tbl:cldd0:sscc0:01}, 
nominally a table of library variants, also indicates which 
processors and tool chains are supported.  

\begin{table}
\caption{\emph{\productbasenameshort{}} Library Variants}
\label{tbl:cldd0:sscc0:01}
\begin{center}
\begin{tabular}{|c|l|l|l|}
\hline
Library & Processor & Compiler, Compiler Options, ABI                         & Remarks                               \\
Variant & Variant   &                                                         &                                       \\
\hline
\hline
   1    & x64       & \emph{Microsoft Visual Studio}                          &                                       \\
        &           & (\index{Windows@\emph{Windows}}\emph{Windows} only)     &                                       \\
        &           & Default compiler options, default ABI                   &                                       \\
\hline
   2    & x64       & \emph{gcc} for \index{----nix@\emph{*nix}}\emph{*nix}   &                                       \\
        &           & Default compiler options, default ABI                   &                                       \\
\hline
   3    & PPC       & \emph{GHS} for \emph{PPC}                               &                                       \\
        & Book      & (only \emph{Windows} supported)                         &                                       \\
        & VLE       & Default compiler options, default ABI                   &                                       \\
\hline
\end{tabular}
\end{center}
\end{table}

The library is not purely generic C code, and can't be 
compiled for unsupported platforms, for two reasons.  First, 
the library source code is generated from templates on a 
per-library-variant basis, (\S{}\ref{cldd0:sgst0}, p.  
\pageref{cldd0:sgst0}; \S{}\ref{cbpc0:sgsc0}, p.  
\pageref{cbpc0:sgsc0}), so generating the library source 
code for unsupported platforms requires modifying both the 
generator and the templates.  Second, a small number of 
library functions are partially or fully coded in 
processor/\-tool/\-ABI-specific assembly-language 
(\S{}\ref{cldd0:salf0}, p.  \pageref{cldd0:salf0}), and no 
generic \emph{C}/\emph{C++} equivalent is provided.


%%%%%%%%%%%%%%%%%%%%%%%%%%%%%%%%%%%%%%%%%%%%%%%%%%%%%%%%%%%%%%%%%%%%%%%%%%%%%%%
%%%%%%%%%%%%%%%%%%%%%%%%%%%%%%%%%%%%%%%%%%%%%%%%%%%%%%%%%%%%%%%%%%%%%%%%%%%%%%%
%%%%%%%%%%%%%%%%%%%%%%%%%%%%%%%%%%%%%%%%%%%%%%%%%%%%%%%%%%%%%%%%%%%%%%%%%%%%%%%
\section{Minimal Footprint in the Embedded Software}
\label{cldd0:smfe0}

\emph{\productbasenameshort{}} is arranged as a ``true'' (or 
``traditional'') library, where the linker extracts only the 
functions and data necessary to support the use of the 
library by the client software.  The 
\emph{\productbasenameshort{}} footprint in the embedded 
software will be automatically minimized by using one 
translation unit per function or data structure, and by the 
nature of the library linking process.  

Some linkers scan a library multiple times until no new 
symbols are resolved (so that the ordering of object files 
within the library doesn't matter), but some require 
references to symbols to come before definitions.  Rather 
than manually specifying the order of object files, the 
templating language supports ``requires'' and ``provides'' 
information so that the library creation process 
automatically includes object modules in a suitable 
order.\footnote{This also provides a calling tree for the 
library.} 


%%%%%%%%%%%%%%%%%%%%%%%%%%%%%%%%%%%%%%%%%%%%%%%%%%%%%%%%%%%%%%%%%%%%%%%%%%%%%%%
%%%%%%%%%%%%%%%%%%%%%%%%%%%%%%%%%%%%%%%%%%%%%%%%%%%%%%%%%%%%%%%%%%%%%%%%%%%%%%%
%%%%%%%%%%%%%%%%%%%%%%%%%%%%%%%%%%%%%%%%%%%%%%%%%%%%%%%%%%%%%%%%%%%%%%%%%%%%%%%
\section{Memory Allocation Paradigm}
\label{cldd0:smap0}

\emph{\productbasenameshort{}} contains no calls to dynamic 
memory allocation functions.  This is consistent with 
standard practice in embedded systems.  

All RAM used by \emph{\productbasenameshort{}} is either 
storage class \emph{auto} (allocated on the stack), storage 
class \emph{register} (functionally the same as 
\emph{auto}), or part of a buffer area provided by the 
caller.  

\emph{\productbasenameshort{}} allocates small variables 
(integers, pointers) as \emph{auto}\@.  Larger variables 
(large integer inputs, large integer outputs, cryptographic 
hash state, etc.) are in memory areas provided by the 
caller.  

In embedded computing environments where stack is limited, 
this strategy allows the client software to decide the best 
place to locate large memory blocks (and the client software 
may still allocate large memory blocks on the stack and 
provide these to the called \emph{\productbasenameshort{}} 
function if appropriate for the computing environment).  

This strategy also pushes thread-safety and core-safety 
issues up to the client software: 

\begin{itemize}
\item \emph{auto} storage is always thread-safe and
      core-safe.
\item Since the memory for input parameters, output
      parameters, and large temporary storage always
      belongs to the caller, the caller must deal
      with any thread-safety and core-safety issues.
\end{itemize}


%%%%%%%%%%%%%%%%%%%%%%%%%%%%%%%%%%%%%%%%%%%%%%%%%%%%%%%%%%%%%%%%%%%%%%%%%%%%%%%
%%%%%%%%%%%%%%%%%%%%%%%%%%%%%%%%%%%%%%%%%%%%%%%%%%%%%%%%%%%%%%%%%%%%%%%%%%%%%%%
%%%%%%%%%%%%%%%%%%%%%%%%%%%%%%%%%%%%%%%%%%%%%%%%%%%%%%%%%%%%%%%%%%%%%%%%%%%%%%%
\section{Security of the Library}
\label{cldd0:sstl0}

Need to resolve the following questions:

Open questions:

\begin{itemize}
\item Are there any security standards or best practices
      that apply to cryptographic implementations?
\item Are there any requirements or best practices for 
      clearing memory? (I've noticed
      that some implementations clear memory of
      keys as soon as they
      are no longer required.  This is obvious.
      But it also might be
      possible to leak information via stack frame
      contents that persist
      after functions return.  Any standards for
      clearing this?)
\item How does one clear variables that are \emph{auto}?
      Setting the variable at the end of a function won't 
      have any effect, because any optimizing compiler will 
      remove the statements.  Declaring variables 
      \emph{volatile}?  Using some sort of a cast involving 
      a pointer to something volatile?
\item Any techniques for resistance to side channel 
      attacks?
\item Any other security best practices?
\end{itemize}

Need to get this settled in terms of design and interfaces.
And best practices.
The two most attractive and obvious approaches are:

\begin{itemize}
\item Pass a parameter in to indicate whether security
      is required.
\item Provide both secure and insecure functions.  (The
      FLASH bloat may not be as bad as one would think,
      because many secure functions could possibly
      be 
      expressed as calls to insecure functions with 
      additional memory security added.)
\end{itemize}

The first approach is likely the more attractive.


%%%%%%%%%%%%%%%%%%%%%%%%%%%%%%%%%%%%%%%%%%%%%%%%%%%%%%%%%%%%%%%%%%%%%%%%%%%%%%%
%%%%%%%%%%%%%%%%%%%%%%%%%%%%%%%%%%%%%%%%%%%%%%%%%%%%%%%%%%%%%%%%%%%%%%%%%%%%%%%
%%%%%%%%%%%%%%%%%%%%%%%%%%%%%%%%%%%%%%%%%%%%%%%%%%%%%%%%%%%%%%%%%%%%%%%%%%%%%%%
\section{Thread Safety}
\label{cldd0:stsf0}

The notion of \index{thread-safe}thread safety is discussed 
in \S{}\ref{cgct0:sipc0} (p.  \pageref{cgct0:sipc0}).  

All functions in \emph{\productbasenameshort{}} are 
thread-safe in that they can be safely called from multiple 
threads running on the same core, with the following notes 
and restrictions: 

\begin{itemize}
\item The memory areas supplied by the client software
      can't be read or written by other threads in a 
      conflicting way until after the
      \emph{\productbasenameshort{}} function has returned
      (it is the client software's responsibility to ensure 
      this).
\item Memory barriers aren't necessary before or after calls
      into \emph{\productbasenameshort{}}\@.  (The reason is 
      that all processors I'm aware of ensure that any 
      stream of instructions, including one that
      occurs as the result of an interrupt, is handled
      correctly.  An interrupt is necessary to switch to 
      another thread, i.e. the switch can't happen
      \emph{without} an interrupt.)
\item Compiler barriers aren't necessary before or after
      calls into \emph{\productbasenameshort{}}\@.  (The  
      reason is that \index{inline
      function@\emph{inline} function}\emph{inline} 
      interface functions are prohbited by 
      \emph{\productbasenameshort{}} design rules, so all
      calls to \emph{\productbasenameshort{}} function
      are true calls.  The compiler is not free to 
      reorganize statements beyond a function
      boundary, so the packaging of a function creates
      inherent compiler barriers.)
\end{itemize}

All functions in \emph{\productbasenameshort{}} can also 
generally be called from an interrupt context, so long as 
the restrictions above are met and so long as the way the 
ISR is wrapped supports the ABI assumptions\@.  (ABI 
adherence is required because it would be possible to 
accidentally call a \emph{\productbasenameshort{}} function 
that modifies a CPU register not saved and restored by the 
ISR wrapper.) 


%%%%%%%%%%%%%%%%%%%%%%%%%%%%%%%%%%%%%%%%%%%%%%%%%%%%%%%%%%%%%%%%%%%%%%%%%%%%%%%
%%%%%%%%%%%%%%%%%%%%%%%%%%%%%%%%%%%%%%%%%%%%%%%%%%%%%%%%%%%%%%%%%%%%%%%%%%%%%%%
%%%%%%%%%%%%%%%%%%%%%%%%%%%%%%%%%%%%%%%%%%%%%%%%%%%%%%%%%%%%%%%%%%%%%%%%%%%%%%%
\section{Core Safety}
\label{cldd0:scsf0}

By \index{core-safe}\emph{core-safe}, I mean that a function 
can be safely used by multiple processor cores in an 
SMP\footnote{The term \emph{SMP} is misused here, as 
described in \S{}\ref{ciov0:snom0:sptr0}, p.  
\pageref{ciov0:snom0:sptr0}.} system, with only one copy of 
the function contained in FLASH memory.  

All functions in \emph{\productbasenameshort{}} are core-safe,
so long as the following conditions are met:

\begin{itemize}
\item The restrictions discussed for thread safety in
      \S{}\ref{cldd0:stsf0} (p.  \pageref{cldd0:stsf0})
      are met, both for ordinary and interrupt context.
\item Some mechanism is used to ensure that multiple cores
      cannot access the same input, input/output, output, or 
      scratchpad memory at the same time.\footnote{The
      assumption is made that the stacks allocated to each
      process on each core are separate so that \emph{auto} 
      and \emph{register} variables are distinct.  No 
      software architecture I'm aware of violates this
      assumption.}
\item Memory barriers, cache line flushes, and cache line
      invalidations are used as necessary in the
      coordination between cores (see
      \S{}\ref{cgct0:sooe0}, p. \pageref{cgct0:sooe0};
      \S{}\ref{cgct0:scco0}, p. \pageref{cgct0:scco0}).
      For example, assume
      that \emph{Core} $\alpha$ calls a function in
      \emph{\productbasenameshort{}} then sets a flag
      $f$ to \emph{true} to
      signal to \emph{Core} $\beta$ that data from
      the function call is ready.
      \begin{itemize}
      \item On the part of \emph{Core} $\alpha$, after the
            call into \emph{\productbasenameshort{}}, the 
            following steps may be required:
            \begin{itemize}
            \item A memory barrier instruction to ensure
                  that there
                  are no lingering
                  writes not completed from the call into
                  \emph{\productbasenameshort{}}.
            \item A cache line flush instruction to ensure
                  that the data
                  from the function
                  call is written to RAM.
            \item Set \emph{f} to \emph{true}.
            \item A cache line flush instruction to ensure
                  that the changed value of
                  \emph{f} is written to RAM.
            \end{itemize}
      \item On the part of \emph{Core} $\beta$, to receive
            the data,
            the following steps may be required:
            \begin{itemize}
            \item A cache line invalidation instruction to
                  invalidate the
                  cache line containing $f$.
            \item A test of $f$.
            \item If $f$ is \emph{true}:
                  \begin{itemize}
                  \item A cache line invalidation
                        instruction to invalidate
                        the cache line containing the data.
                  \item Use of the data.
                  \end{itemize}
            \end{itemize}
      \end{itemize}
\end{itemize}

The approach taken by \emph{\productbasenameshort{}} is to 
allow the client software to handle threading and core 
coordination issues.  Handling these issues is potentially 
complex (as illustrated by the example above).  


%%%%%%%%%%%%%%%%%%%%%%%%%%%%%%%%%%%%%%%%%%%%%%%%%%%%%%%%%%%%%%%%%%%%%%%%%%%%%%%
%%%%%%%%%%%%%%%%%%%%%%%%%%%%%%%%%%%%%%%%%%%%%%%%%%%%%%%%%%%%%%%%%%%%%%%%%%%%%%%
%%%%%%%%%%%%%%%%%%%%%%%%%%%%%%%%%%%%%%%%%%%%%%%%%%%%%%%%%%%%%%%%%%%%%%%%%%%%%%%
\section{Representation of Numbers}
\label{cldd0:srnm0}


%%%%%%%%%%%%%%%%%%%%%%%%%%%%%%%%%%%%%%%%%%%%%%%%%%%%%%%%%%%%%%%%%%%%%%%%%%%%%%%
%%%%%%%%%%%%%%%%%%%%%%%%%%%%%%%%%%%%%%%%%%%%%%%%%%%%%%%%%%%%%%%%%%%%%%%%%%%%%%%
%%%%%%%%%%%%%%%%%%%%%%%%%%%%%%%%%%%%%%%%%%%%%%%%%%%%%%%%%%%%%%%%%%%%%%%%%%%%%%%
\subsection{Native Unsigned and Signed Integers}
\label{cldd0:srnm0:srin0}

In all cases, \emph{\productbasenameshort{}} represents 
native integers in the same way as the compiler; and the 
compiler, in turn, invariably represents native integers in 
the same way as the processor.  

Although the \emph{C}/\emph{C++} language allows 
representations of signed integers other than two's 
complement, \emph{\productbasenameshort{}} will support 
only\footnote{Representation assumptions always creep into 
code when the goal is to be efficient.  I'm not aware of any 
modern microcontroller that \emph{doesn't} use two's 
complement representation for native integers.} processors 
that use two's complement representation of signed integers.  


%%%%%%%%%%%%%%%%%%%%%%%%%%%%%%%%%%%%%%%%%%%%%%%%%%%%%%%%%%%%%%%%%%%%%%%%%%%%%%%
%%%%%%%%%%%%%%%%%%%%%%%%%%%%%%%%%%%%%%%%%%%%%%%%%%%%%%%%%%%%%%%%%%%%%%%%%%%%%%%
%%%%%%%%%%%%%%%%%%%%%%%%%%%%%%%%%%%%%%%%%%%%%%%%%%%%%%%%%%%%%%%%%%%%%%%%%%%%%%%
\subsection{Large Unsigned Integers}
\label{cldd0:srnm0:srlu0}

Efficient large integer arithmetic has been studied 
extensively.  Efficient arithmetic on large integer operands 
relies on using the machine instructions available for small 
integer operands repeatedly in order to operate on large 
integer operands.  To quote and paraphrase Knuth \cite[pp.  
265-266]{bibref:b:TaocpVolume2}: 

\begin{em}
\begin{quote}
The most important fact to understand about 
extended-precision numbers is that they may be regarded as 
numbers written in radix $w$ notation, where $w$ is the 
computer's word size.  For example, an integer that fills 10 
words on a computer whose word size is $w=10^{10}$ has 100 
decimal digits; but we will consider it to be a 10-place 
number to the base $10^{10}$.  This viewpoint is justified 
for the same reason that we may convert, say, from binary to 
hexadecimal notation, simply by grouping the bits together.  

In these terms, we are given the following primitive 
operations to work with: 

\begin{enumerate}
\item \label{cldd0:snom0:scan0:e1:n1}addition or subtraction
      of one-place integers, giving a one-place answer and a 
      carry;
\item \label{cldd0:snom0:scan0:e1:n2}multiplication of a
      one-place integer by another one-place integer, giving 
      a two-place answer;
\item \label{cldd0:snom0:scan0:e1:n3}division of a two-place
      integer by a one-place integer, provided that the 
      quotient is a one-place integer, and yielding also a 
      one-place remainder.
\end{enumerate}

By adjusting the word size, if necessary, nearly all 
computers will have these three operations available; so we 
will construct algorithms for addition, subtraction, 
multiplication, and division of arbitrary-size integers in 
terms of the primitive operations 
(\ref{cldd0:snom0:scan0:e1:n1}), 
(\ref{cldd0:snom0:scan0:e1:n2}), and 
(\ref{cldd0:snom0:scan0:e1:n3}).  
\end{quote}
\end{em}

Processors may vary from Knuth's description above in two ways:

\begin{itemize}
\item Some low-end processors do not have some instruction
      types at all.  (Among 8-bit microcontrollers, for 
      example, some do not have an integer division
      instruction, some do not have a integer multipication
      instruction, and
      some do not have either.)
\item Some instruction sets don't match\footnote{For
      example, a processor may have a 32-bit integer 
      multiplication instruction that produces a
      64-bit result, but not have a corresponding integer
      division instruction
      that accepts a 64-bit divisor.} Knuth's description
      in \cite[pp.
      265-266]{bibref:b:TaocpVolume2}, so that the natural
      limb size for some
      arithmetic operations may be different than for other
      arithmetic
      operations.  In these cases, deciding on a limb size
      may be subjective or
      involve some inelegant code to compensate for the
      irregular instruction
      set.
\end{itemize}

%Note:  GMP Library \index'd outside of footnote below to
%       work around LaTeX issue.  See description at end of 
%       file.
I define \index{limb size}\emph{limb size}\index{GMP 
Library@\emph{GMP Library}}\footnote{The terms \emph{limb} 
and \emph{limb size} come from \emph{The GMP Library} 
\cite{bibref:w:gmplibhomepage}.} as the number of bits in 
what Knuth calls $w$ (from the quote above).  

In \emph{\productbasenameshort{}}, large integers are 
represented as arrays of what I call \emph{limbs}, where a 
limb is of the \emph{C}/\emph{C++} native data type that 
contains \emph{limb size} bits and is the integral native 
data type most suited for the purpose.\footnote{Although it 
varies across processors and compilers, a typical limb size 
is 32 bits, and a typical limb is a \emph{C}/\emph{C++} 
\emph{unsigned int}.} The least significant limb is stored 
first, as \emph{C}/\emph{C++} array element 
[0].\footnote{For humans, this is perhaps a counterintuitive 
order.  However, this order simplifies functions that 
perform arithmetic limb-by-limb, and is the de facto 
standard order in software that implements large integer 
arithmetic.} 

Due to endianness issues, if large integers are represented 
as arrays of limbs (where each limb is larger than one 
byte), there is no portable way to access these large 
integers as an array of another data type (as an array of 
\emph{unsigned char}s, for example).  


%%%%%%%%%%%%%%%%%%%%%%%%%%%%%%%%%%%%%%%%%%%%%%%%%%%%%%%%%%%%%%%%%%%%%%%%%%%%%%%
%%%%%%%%%%%%%%%%%%%%%%%%%%%%%%%%%%%%%%%%%%%%%%%%%%%%%%%%%%%%%%%%%%%%%%%%%%%%%%%
%%%%%%%%%%%%%%%%%%%%%%%%%%%%%%%%%%%%%%%%%%%%%%%%%%%%%%%%%%%%%%%%%%%%%%%%%%%%%%%
\subsection{Large Signed Integers}
\label{cldd0:srnm0:srls0}

Conceptually, a large signed integer could be represented in 
\index{two's complement}two's complement form, with the 
number distributed over an array of limbs.  If that were 
done, addition, subtraction, and comparison would be 
efficient.  

However, the problems introduced by a two's complement 
representation of large signed integers would be: 

\begin{itemize}
\item Multiplication and division would be awkward and
      potentially require negating the large signed integer
      before the
      operation.
\item Negating a large signed integer takes time.
\item Negating a large signed integer requires additional
      memory\@.  (Because
      the input parameters to \emph{\productbasenameshort{}}
      are typically
      constants passed by reference, they can't be negated
      in place.)
\end{itemize}

Because of the problems associated with a two's complement 
representation of large signed integers, the representation 
used by \emph{\productbasenameshort{}} is 
\index{sign-magnitude}\emph{sign-magnitude}.  The most 
significant bit of the most significant limb is reserved as 
a sign bit, and all other bits contain the magnitude.  A 
value of 0 for the sign bit implies zero or a positive 
number, and a value of 1 for the sign bit implies a negative 
number.  

This representation has the disadvantage that there are two 
representations of 0---one with the sign bit clear and one 
with the sign bit set (for brevity, I will call these two 
representations $+0$ and $-0$, respectively).  

To minimize the repercussions of two representations of 0, 
\emph{\productbasenameshort{}} has the following guaranteed 
behavior: 

\begin{itemize}
\item Canonically, $+0$ is defined to be the only proper
      representation of 0.
\item \emph{\productbasenameshort{}} is written so that
      $-0$ as input to a
      function will always
      produce the same result as $+0$.\footnote{However,
      there
      is no guarantee made of equal execution time,
      although any difference
      will be small.}
\item In any context where \emph{\productbasenameshort{}}
      produces a result
      of 0, it will produce $+0$.
\end{itemize}


%%%%%%%%%%%%%%%%%%%%%%%%%%%%%%%%%%%%%%%%%%%%%%%%%%%%%%%%%%%%%%%%%%%%%%%%%%%%%%%%
%%%%%%%%%%%%%%%%%%%%%%%%%%%%%%%%%%%%%%%%%%%%%%%%%%%%%%%%%%%%%%%%%%%%%%%%%%%%%%%%
%%%%%%%%%%%%%%%%%%%%%%%%%%%%%%%%%%%%%%%%%%%%%%%%%%%%%%%%%%%%%%%%%%%%%%%%%%%%%%%%
\subsection{Native Unsigned and Signed Fixed-Point Numbers}
\label{cldd0:srnm0:snfp0}

Native fixed point numbers are represented in the same way as native
integers (\S{}\ref{cldd0:srnm0:srin0}).  $k$ is always chosen to be a
power of two in order to place the radix point between two bits.


%%%%%%%%%%%%%%%%%%%%%%%%%%%%%%%%%%%%%%%%%%%%%%%%%%%%%%%%%%%%%%%%%%%%%%%%%%%%%%%%
%%%%%%%%%%%%%%%%%%%%%%%%%%%%%%%%%%%%%%%%%%%%%%%%%%%%%%%%%%%%%%%%%%%%%%%%%%%%%%%%
%%%%%%%%%%%%%%%%%%%%%%%%%%%%%%%%%%%%%%%%%%%%%%%%%%%%%%%%%%%%%%%%%%%%%%%%%%%%%%%%
\subsection{Large Unsigned Fixed-Point Numbers}
\label{cldd0:srnm0:srlu5}

Large unsigned fixed point numbers are represented in the same way as
large unsigned integers (\S{}\ref{cldd0:srnm0:srlu0}).  $k$ is always
chosen to be a power of two in order to place the radix point between two
bits.


%%%%%%%%%%%%%%%%%%%%%%%%%%%%%%%%%%%%%%%%%%%%%%%%%%%%%%%%%%%%%%%%%%%%%%%%%%%%%%%%
%%%%%%%%%%%%%%%%%%%%%%%%%%%%%%%%%%%%%%%%%%%%%%%%%%%%%%%%%%%%%%%%%%%%%%%%%%%%%%%%
%%%%%%%%%%%%%%%%%%%%%%%%%%%%%%%%%%%%%%%%%%%%%%%%%%%%%%%%%%%%%%%%%%%%%%%%%%%%%%%%
\subsection{Large Signed Fixed-Point Numbers}
\label{cldd0:srnm0:srlu6}

Large signed fixed point numbers are represented in the same 
way as large signed integers (\S{}\ref{cldd0:srnm0:srls0}).  
$k$ is always chosen to be a power of two in order to place 
the radix point between two bits.  


%%%%%%%%%%%%%%%%%%%%%%%%%%%%%%%%%%%%%%%%%%%%%%%%%%%%%%%%%%%%%%%%%%%%%%%%%%%%%%%%
%%%%%%%%%%%%%%%%%%%%%%%%%%%%%%%%%%%%%%%%%%%%%%%%%%%%%%%%%%%%%%%%%%%%%%%%%%%%%%%%
%%%%%%%%%%%%%%%%%%%%%%%%%%%%%%%%%%%%%%%%%%%%%%%%%%%%%%%%%%%%%%%%%%%%%%%%%%%%%%%%
\subsection{Native Floating-Point Numbers}
\label{cldd0:srnm0:sfpn0}

No decision has been made yet on the representation of 
native floating-point numbers (there are too many open 
questions, below).  

My original position was that floating-point numbers aren't 
useful in embedded computing, primarily due to the high cost 
of performing floating-point arithmetic in software.  
However, \index{Davidson, Eileen A.}Eileen Davidson 
\cite{bibref:i:EileenADavidson} pointed out that many modern 
microcontrollers implement floating-point arithmetic in 
hardware, so that efficiency concerns may not apply.  

Questions:

\begin{itemize}
\item How fast is a typical FPU:
      \begin{itemize}
      \item Compared to integer operations, including
            effects on the
            pipeline?
      \item Compared to software implementation of the
            floating-point
            operations?
      \end{itemize}
\item What data types are accepted?  (I remember some older
      FPU's would handle the equivalent of both
      floats and doubles.)  The answer to this question may
      imply that there is more than one native
      representation.
\item Are the data types covered by a standard (perhaps an
      IEEE standard)?
\item Do whatever standards are adhered to require an
      exact and defined answer to any floating-point
      operation?  The answer to this question has two
      implications:
      \begin{itemize}
      \item Precise accuracy requirements may help to
            rigorously verify certain mathematical
            algorithms.
      \item If two different FPU's from different sources
            aren't required to generate the same result,
            down to the bit, testing on one platform may
            have no bearing on testing on another platform
            (unlike integers and integer operations, which
            are exactly defined).
      \end{itemize}
\end{itemize}


%%%%%%%%%%%%%%%%%%%%%%%%%%%%%%%%%%%%%%%%%%%%%%%%%%%%%%%%%%%%%%%%%%%%%%%%%%%%%%%%
%%%%%%%%%%%%%%%%%%%%%%%%%%%%%%%%%%%%%%%%%%%%%%%%%%%%%%%%%%%%%%%%%%%%%%%%%%%%%%%%
%%%%%%%%%%%%%%%%%%%%%%%%%%%%%%%%%%%%%%%%%%%%%%%%%%%%%%%%%%%%%%%%%%%%%%%%%%%%%%%%
\subsection{Large Floating-Point Numbers}
\label{cldd0:srnm0:sfpl0}

It isn't clear if there is any meaningful notion of a 
``large'' or ``extended-precision'' floating-point number 
that has any usefulness in microcontroller work.  

Questions:

\begin{itemize}
\item How does \emph{The GMP Library} implement
      arbitrary-precision floating-point numbers (there are 
      probably some lessons to be learned there)?
\item Are there any standards for large
      floating-point numbers?
\item Can an FPU be used to assist in the calculation of
      floating-point operands larger than it can handle?
\item Can an FPU be used to assist with integer
      calculations?
\item Can integer instructions be used to assist with large
      floating-point operations?
\end{itemize}

The answers to the last three questions aren't obvious at 
all without more investigation.  The best known algorithm 
for large integer division involves using the built-in 
\emph{DIV} instruction to estimate the next word of the 
quotient.  This may be counterintuitive.  The best 
algorithms for large floating-point operations may also be 
counterintuitive.  


%%%%%%%%%%%%%%%%%%%%%%%%%%%%%%%%%%%%%%%%%%%%%%%%%%%%%%%%%%%%%%%%%%%%%%%%%%%%%%%
%%%%%%%%%%%%%%%%%%%%%%%%%%%%%%%%%%%%%%%%%%%%%%%%%%%%%%%%%%%%%%%%%%%%%%%%%%%%%%%
%%%%%%%%%%%%%%%%%%%%%%%%%%%%%%%%%%%%%%%%%%%%%%%%%%%%%%%%%%%%%%%%%%%%%%%%%%%%%%%
\section{Generation of Source Code from Templates}
\label{cldd0:sgst0}

To keep the library source code uncluttered for review, unit 
testing, and debugging, the source code is generated on a 
per-library-variant basis by a generator, which is a generic 
C program (\S{}\ref{cbpc0:sgsc0}, p.  
\pageref{cbpc0:sgsc0}).  This results in unique source code 
for each processor / tool / ABI combination.  

After generation, the source code that is used to produce a 
library has preprocessor switches only for tuning the 
library (\S{}\ref{cbpc0:sltl0}, p.  \pageref{cbpc0:sltl0}).  
All other variability is removed from the source code and 
build process by the generation process.  


%%%%%%%%%%%%%%%%%%%%%%%%%%%%%%%%%%%%%%%%%%%%%%%%%%%%%%%%%%%%%%%%%%%%%%%%%%%%%%%
%%%%%%%%%%%%%%%%%%%%%%%%%%%%%%%%%%%%%%%%%%%%%%%%%%%%%%%%%%%%%%%%%%%%%%%%%%%%%%%
%%%%%%%%%%%%%%%%%%%%%%%%%%%%%%%%%%%%%%%%%%%%%%%%%%%%%%%%%%%%%%%%%%%%%%%%%%%%%%%
\section{Assembly-Language Functions}
\label{cldd0:salf0}

Certain \emph{\productbasenameshort{}} functions and 
portions of functions are written in assembly-language.  The 
assembly-language is potentially 
pro\-ces\-sor-var\-i\-ant-spe\-cif\-ic, tool-spe\-cif\-ic, 
and ABI-spe\-cif\-ic.  

Assembly-language for some low-level operations is necessary 
in numerical and cryptographic libraries.  


%%%%%%%%%%%%%%%%%%%%%%%%%%%%%%%%%%%%%%%%%%%%%%%%%%%%%%%%%%%%%%%%%%%%%%%%%%%%%%%
%%%%%%%%%%%%%%%%%%%%%%%%%%%%%%%%%%%%%%%%%%%%%%%%%%%%%%%%%%%%%%%%%%%%%%%%%%%%%%%
%%%%%%%%%%%%%%%%%%%%%%%%%%%%%%%%%%%%%%%%%%%%%%%%%%%%%%%%%%%%%%%%%%%%%%%%%%%%%%%
\section{Supported Versions of the \emph{C}/\emph{C++} Language}
\label{cldd0:sscv0}

This will have to be codified, but I don't expect any 
difficulties with any modern \emph{C} or \emph{C++} 
compilers.  The \emph{C++} language standard specifies a 
preprocessor symbol that is defined only when using a 
\emph{C++} compiler, and this can be used to conditionally 
insert ``\texttt{extern "C"}'' in the compile stream (which 
causes a \emph{C++} compiler to use \emph{C} function naming 
and calling conventions, i.e.  doesn't use the name mangling 
scheme).  I don't see any reason that the library can't be 
used easily from any modern version of \emph{C}/\emph{C++}.  


%%%%%%%%%%%%%%%%%%%%%%%%%%%%%%%%%%%%%%%%%%%%%%%%%%%%%%%%%%%%%%%%%%%%%%%%%%%%%%%
%%%%%%%%%%%%%%%%%%%%%%%%%%%%%%%%%%%%%%%%%%%%%%%%%%%%%%%%%%%%%%%%%%%%%%%%%%%%%%%
%%%%%%%%%%%%%%%%%%%%%%%%%%%%%%%%%%%%%%%%%%%%%%%%%%%%%%%%%%%%%%%%%%%%%%%%%%%%%%%
\section{Formal Parameters and Return Values of Functions}
\label{cldd0:sfcp0}

In all cases, the formal parameters and return value of a 
\emph{\productbasenameshort{}} interface function will take 
exactly one of two forms, \emph{simple} or \emph{extended}, 
described below.  

The purpose of imposing design rules on the formal 
parameters and return value is to simplify the use of the 
library by restricting the \emph{\productbasenameshort{}} 
interface styles.  

The design rules presented here do not apply to 
\emph{internal} interfaces in 
\emph{\productbasenameshort{}}\@.  They apply only to 
interfaces between client software and 
\emph{\productbasenameshort{}}.  


%%%%%%%%%%%%%%%%%%%%%%%%%%%%%%%%%%%%%%%%%%%%%%%%%%%%%%%%%%%%%%%%%%%%%%%%%%%%%%%
%%%%%%%%%%%%%%%%%%%%%%%%%%%%%%%%%%%%%%%%%%%%%%%%%%%%%%%%%%%%%%%%%%%%%%%%%%%%%%%
%%%%%%%%%%%%%%%%%%%%%%%%%%%%%%%%%%%%%%%%%%%%%%%%%%%%%%%%%%%%%%%%%%%%%%%%%%%%%%%
\subsection{Simple Interface Style}
\label{cldd0:sfcp0:sifs0}

The \emph{simple interface style} can be used if the 
following conditions are met: 

\begin{itemize}
\item Function input(s) consist of only \emph{C}/\emph{C++}
      native data types, \emph{and}
\item All function inputs are passed by value (not by
      reference), \emph{and}
\item The function returns a \emph{single} output, 
      a \emph{C}/\emph{C++}
      native data type, \emph{and}
\item The function does not return a result code with
      success/failure
      information, but instead returns only data, \emph{and}
\item The function does not require any temporary memory
      allocated by the client software.
\end{itemize}

A typical function using a simple interface style might be 
something like \texttt{unsigned NlUiSqrt(unsigned arg)} or 
\texttt{unsigned NlUiMax2(unsigned arg1, unsigned arg2)}.  


%%%%%%%%%%%%%%%%%%%%%%%%%%%%%%%%%%%%%%%%%%%%%%%%%%%%%%%%%%%%%%%%%%%%%%%%%%%%%%%
%%%%%%%%%%%%%%%%%%%%%%%%%%%%%%%%%%%%%%%%%%%%%%%%%%%%%%%%%%%%%%%%%%%%%%%%%%%%%%%
%%%%%%%%%%%%%%%%%%%%%%%%%%%%%%%%%%%%%%%%%%%%%%%%%%%%%%%%%%%%%%%%%%%%%%%%%%%%%%%
\subsection{Extended Interface Style}
\label{cldd0:sfcp0:sefs0}

The extended interface style must be used if any of the
following conditions are met:

\begin{itemize}
\item Function input(s) involve non-native data types,
      \emph{or}
\item Any function inputs are passed by reference (as might
      be natural for large inputs where pushing them on the 
      stack would be ill-advised), \emph{or}
\item The function returns more than one output, \emph{or}
\item The function returns an output that is not a
      \emph{C}/\emph{C++} native data type, \emph{or}
\item The function returns a result code with
      success/failure information, \emph{or}
\item The function requires temporary memory
      allocated by the client software.
\end{itemize}

A function that uses the the extended interface style has a 
return value and formal parameter list that follows these 
rules: 

\begin{itemize}
\item The function return type must be \emph{void}.
\item The parameter list consists, in order, of:
      \begin{itemize}
      \item $[$Output$]$ One or more status/result codes,
            returned by reference:
            \begin{itemize}
            \item These status/result codes may be:
                  \begin{itemize}
                  \item Boolean values (some condition did
                        or did not occur).
                  \item Bit-masked or bit-packed.
                  \item Enumerated.
                  \item Numerical (number of tasks
                        successfully completed, for 
                        example).
                  \end{itemize}
            \item If allowed by the function documentation,
                  the
                  pointers may be NULL\@.  NULL signifies
                  that the client
                  software isn't interested in the output.
            \end{itemize}
      \item $[$Output$]$ One or more data outputs, returned
            by reference:
            \begin{itemize}
            \item The output values may be of any type.
            \item The memory areas supplied for different
                  formal parameters
                  may never overlap (in some cases, to save
                  memory, the
                  \emph{\productbasenameshort{}} function
                  may use the
                  ``output'' memory areas for temporary
                  storage during the
                  execution of the function, and overlapping
                  areas may
                  cause the \emph{\productbasenameshort{}}
                  function to behave
                  incorrectly).
            \item If allowed by the function documentation,
                  the
                  pointers may be NULL\@.  NULL signifies
                  that the client
                  software isn't interested in the output.
            \end{itemize}
      \item $[$Input/Output$]$ One or more data
            inputs/outputs, passed and
            returned by reference:
            \begin{itemize}
            \item The input/output values may be of any
                  type.
            \item The memory areas supplied for different
                  formal parameters
                  may never overlap (in some cases, the
                  \emph{\productbasenameshort{}} function
                  may use the
                  input/output memory areas for temporary
                  storage during the
                  execution of the function, and overlapping
                  areas may
                  cause the \emph{\productbasenameshort{}}
                  function to behave
                  incorrectly).
            \item If allowed by the function documentation,
                  the
                  pointers may be NULL\@.  NULL signifies
                  that the client
                  software doesn't provide this input and/or
                  isn't interested in this
                  output.
            \end{itemize}
      \item $[$Input$]$ One or more data inputs, passed by
            value or by reference:
            \begin{itemize}
            \item The input values may be of any type.
            \item In the case of data passed by reference,
                  it is guaranteed
                  that the \emph{\productbasenameshort{}}
                  function
                  will never execute a write to the memory
                  area containing
                  the input.
            \item The input values will typically be
                  declared \texttt{const *}.
            \item The input memory areas may overlap with
                  each other, but may not overlap
                  with input/output, output, or scratchpad
                  memory parameters.
            \item If allowed by the function documentation,
                  the
                  pointers may be NULL\@.  NULL signifies
                  that the client
                  software doesn't provide this input.
            \end{itemize}
      \item $[$Scratchpad$]$\footnote{Yes, I'm looking for
            a better name to describe this
            type of parameter.}  One or more memory areas,
            passed by reference:
            \begin{itemize}
            \item Will be used by the
                  \emph{\productbasenameshort{}} function 
                  for
                  ``large'' temporary storage that is only
                  required while the
                  function executes.
            \item In some cases, might be allocated from
                  the stack by the client software.
            \item Do not need to be initialized in any way
                  before being used in
                  \emph{\productbasenameshort{}} function
                  calls.
            \item Values need not be retained by the client
                  software after the 
                  \emph{\productbasenameshort{}} function
                  returns.
            \item No assumptions can be made by the client
                  software about the contents
                  after the \emph{\productbasenameshort{}}
                  function returns.
            \item Memory blocks will have a ``* struct''
                  type, so that the client software
                  allocates the right amount of memory
                  just by defining a variable.
            \item In many cases, the client software would
                  likely \emph{union} many of these
                  \emph{scratchpad} types together, since
                  they aren't used concurrently (this
                  allows the compiler to select the minimum
                  space required).
            \item These memory areas can't overlap with any
                  other parameters.
            \item If allowed by the function documentation,
                  the
                  pointers may be NULL\@.  NULL signifies 
                  that the client
                  software won't provide this input.
                  However, I would expect that
                  the scratchpad areas are always required
                  if they are ever required.
            \end{itemize}
      \end{itemize}
\end{itemize}

An example combining the ideas for the extended interface 
style may be helpful.  

\begin{verbatim}
#include "NumLib.h" //Function and data structure definitions for //clients of the library.

//Assume that the NumLib Library contains a large unsigned integer
//multiplication function that requires two "scratchpad" areas
//of memory to operate (very contrived).

//Overlay several scratch areas.  This lets the compiler figure
//out largest size.
union
{
   NL_SCRATCH_TYPE_A mul_scratch_a;
      //Must be provided by the caller.  Doesn't matter where from.
      //Here it is global memory.
   NL_DIV_SCRATCH_TYPE_A div_scratch_a;
      //Assume we call a division function somewhere in main() as
      //well.  Since the multiplication and division function can't
      //be active at the same time, the "scratch" areas can be
      //overlaid.
} my_union;

int main(void)
{
   NL_MUL_RESULT mul_result;
   NL_MUL_SCRATCH_TYPE_B mul_scratch_b;
      //Second scratch area is on stack.  This might be done
      //sometimes.

   NL_UI1024 ui1024;
   NL_UI2048 ui2048;

   //Somehow, we initialize ui1024 here.

   //Actually call the multiplication function.  The
   //parameter types come in the order listed above.
   NlU2048U1024U1024UMul
   (
      &mul_result,   //Success/fail and exception codes
                     //filled in by function.
      &ui2048,       //Result of multiplication.
      &ui1024,       //First input.
      &ui1024,       //Second input.  Coincident memory can
                     //only be used with inputs.
      &my_union.mul_scratch_a,
                     //First scratch area to be used during
                     //the function call.  Where this memory
                     //comes from is left to the client
                     //software.
      &mul_scratch_b //Second scratch area to be used during
                     //the function call.  Where this memory
                     //comes from is left to the client
                     //software.
   );

   return 0;
}
\end{verbatim}


%%%%%%%%%%%%%%%%%%%%%%%%%%%%%%%%%%%%%%%%%%%%%%%%%%%%%%%%%%%%%%%%%%%%%%%%%%%%%%
%%%%%%%%%%%%%%%%%%%%%%%%%%%%%%%%%%%%%%%%%%%%%%%%%%%%%%%%%%%%%%%%%%%%%%%%%%%%%%
%%%%%%%%%%%%%%%%%%%%%%%%%%%%%%%%%%%%%%%%%%%%%%%%%%%%%%%%%%%%%%%%%%%%%%%%%%%%%%
\section{Library Versioning}
\label{cldd0:slvn0}

\index{versioning (library)}Version numbers of the library are of the
form \emph{j.m-} (example: \emph{2.5c}) where:

\begin{itemize}
\item \emph{j} is the major version number, a non-negative
      integer.
\item \emph{m} is the minor version number, a non-negative
      integer.
\item \emph{-} is the bug fix version number, a non-negative
      integer.  The bug fix version number
      is represented as a letter in documentation (for
      example, the `c' in `2.5c').
      When represented as an integer, the correspondence
      is that the letter `a' is represented by 0,
      the letter `b' by 1, etc.
\end{itemize}


%%%%%%%%%%%%%%%%%%%%%%%%%%%%%%%%%%%%%%%%%%%%%%%%%%%%%%%%%%%%%%%%%%%%%%%%%%%%%%
%%%%%%%%%%%%%%%%%%%%%%%%%%%%%%%%%%%%%%%%%%%%%%%%%%%%%%%%%%%%%%%%%%%%%%%%%%%%%%
%%%%%%%%%%%%%%%%%%%%%%%%%%%%%%%%%%%%%%%%%%%%%%%%%%%%%%%%%%%%%%%%%%%%%%%%%%%%%%
\section{Library Binary File Naming Conventions}
\label{cldd0:slpn0}

\index{naming conventions (library)}%
A typical binary library file name is 
``\emph{libnumlib\_2\_5c\_1.sm8}''.
In order, the binary library file name contains the
following components:

\begin{itemize}
\item The characters \emph{libnumlib\_} (a constant string
      to identify the library).
\item The version number of the library followed by an
      underscore
      (example: `\emph{2\_5c\_}').
\item The library variant integer identifier, followed by
      an underscore.  (This integer
      is selected from Table \ref{tbl:cldd0:sscc0:01}.)
\item A period followed by a filename extension traditional
      for the
      development tools associated with that library
      variant.
\end{itemize}


%%%%%%%%%%%%%%%%%%%%%%%%%%%%%%%%%%%%%%%%%%%%%%%%%%%%%%%%%%%%%%%%%%%%%%%%%%%%%%
%%%%%%%%%%%%%%%%%%%%%%%%%%%%%%%%%%%%%%%%%%%%%%%%%%%%%%%%%%%%%%%%%%%%%%%%%%%%%%
%%%%%%%%%%%%%%%%%%%%%%%%%%%%%%%%%%%%%%%%%%%%%%%%%%%%%%%%%%%%%%%%%%%%%%%%%%%%%%
\section{Function Naming Conventions}
\label{cldd0:sfnc0}

A typical function name within the general library is 
`\emph{NlLfU16FiltAFiltRxx}'.  Function names within the 
general library usually consist of the following components, 
with some components optional.  

The first component is the string ``\emph{Nl}''\@.  This 
string serves two purposes.  First, it identifies a function 
as a member of the \emph{\productbasenameshort{}}.  Second, 
it helps to prevent global name collisions.  

The second component is two characters representing the 
function family (selected from Table 
\ref{tbl:cldd0:slcv0:sfnc0:03}).  

The third component is a string representing the types of 
the outputs input/outputs, and inputs.  The error/status 
outputs and scratchpad parameters, if any, don't need to be 
cited.  This section of the name serves to prevent name 
collisions among functions that perform the same function 
but on different data types.  For each output, input/output, 
or input; in the same order as the formal parameter list; 
these conventions are used: 

\begin{itemize}
\item For parameters of types that are native C types:
      \begin{itemize}
      \item ``Uc'' for unsigned character.
      \item ``Sc'' for signed character.
      \item ``Us'' for unsigned short integer.
      \item ``Ss'' for signed short integer.
      \item ``Ui'' for integer.
      \item ``Si'' for signed integer.
      \item ``Ul'' for unsigned long integer.
      \item ``Sl'' for signed long integer.
      \item ``Ull'' for unsigned long long integer.
      \item ``Sll'' for signed long long integer.
      \end{itemize}
\item For parameters that are integers of a specific size
      (or, that aren't
      integers but naturally can be represented the same
      way).
      \begin{itemize}
      \item Integers are denoted by `U' (unsigned) or `S'
            (signed), followed by the
            number of bits in the integer
            (for example, ``U1024'').
      \item If the integer is provided in a native data
            type (rather than a
            structure), the string is suffixed with `n'.
            Example:  ``U64n''.
      \end{itemize}
\end{itemize}

For inputs or outputs where there are several of the same 
type, ``\_\underscorepostspace{}n'' may be used to denote 
this compactly.  For example, 
``U1024\_\underscorepostspace{}3'' would indicate that the 
function returns three separate 1024-bit unsigned integers.  
In this notation, inputs, input/outputs, and outputs may not 
be combined.  For example, a function that takes three 
1024-bit unsigned integer inputs and returns two 1024-bit 
unsigned integer outputs would use the string 
``U1024\_\underscorepostspace{}3U1024\_\underscorepostspace{}2'' 
rather than ``U1024\_\underscorepostspace{}5''.  

If a function operates on a characteristic data type in a 
way where there would never be a naming collision if some 
parameters were not incorporated into the naming convention, 
only this characteristic data type needs to be used.  For 
example, ``NlU32FiltLpFo'' might be an acceptable name for 
a first-order low-pass filter that accepts a 32-bit unsigned 
integer as input and provides a 32-bit unsigned integer as 
output.  This name is potentially acceptable because one 
data type (U32) drives multiple parameters, and there would 
not be a naming collision with a similar filter that filters 
other data types.  

The fourth and final component is the function name, with 
possible qualifiers.  Generally, the function name should go 
from general to specific, i.e.  ``FilterSecondOrder'' rather 
than ``SecondOrderFilter''.  

\begin{table}
\caption{\emph{\productbasenameshort{}} Function Family Code}
\label{tbl:cldd0:slcv0:sfnc0:03}
\begin{center}
\begin{tabular}{|l|l|}
\hline
Function       & Meaning                                                                     \\
Family         &                                                                             \\
Code           &                                                                             \\
\hline
\hline
\texttt{Ut}    & Utility or miscellaneous function (Chapter \ref{cnef0}).                    \\
\hline
\texttt{Bm}    & Block memory function (Chapter \ref{cbmf0}).                                \\
\hline
\texttt{Se}    & Search function (Chapter \ref{csea0}).                                      \\
\hline
\texttt{So}    & Sort function (Chapter \ref{csol0}).                                        \\
\hline
\texttt{Am}    & Array manipulation function (Chapter \ref{cami0}).                          \\
\hline
\texttt{Bt}    & Bit-mapped set function (Chapter \ref{cbsf0}).                              \\
\hline
\texttt{Vc}    & Vertical counter function (Chapter \ref{cvco0}).                            \\
\hline
\texttt{Na}    & Native integer utility/arithmetic function (Chapter \ref{cafn0}).           \\
\hline
\texttt{Nm}    & Native integer mathematical function (Chapter \ref{cbaf0}).                 \\
\hline
\texttt{Pa}    & Native fixed-point utility/arithmetic function (Chapter \ref{cfpa0}).       \\
\hline
\texttt{Pm}    & Native fixed-point mathematical function (Chapter \ref{cfpa0}).             \\
\hline
\texttt{Fa}    & Native floating-point utility/arithmetic function (Chapter \ref{caal1}).    \\
\hline
\texttt{Fm}    & Native floating-point mathematical function (Chapter \ref{cafn0}).          \\
\hline
\texttt{Oa}    & Large integer utility/arithmetic function (Chapter \ref{claf0}).            \\
\hline
\texttt{Om}    & Large integer mathematical function (Chapter \ref{claf1}).                  \\
\hline
\texttt{Da}    & Large fixed-point utility/arithmetic function (Chapter \ref{cfpa2}).        \\
\hline
\texttt{Dm}    & Large fixed-point mathematical function (Chapter \ref{cfpa3}).              \\
\hline
\texttt{Ga}    & Large floating-point utility/arithmetic function (Chapter \ref{claf2}).     \\
\hline
\texttt{Gm}    & Large floating-point mathematical function (Chapter \ref{claf3}).           \\
\hline
\texttt{Lf}    & Linear filter and control system functions (Chapter \ref{clfi0}).           \\
\hline
\texttt{Nf}    & Non-linear filter functions (Chapter \ref{cnfi0}).                          \\
\hline
\texttt{Pr}    & Pseudo-random number generation functions (Chapter \ref{crng1}).            \\
\hline
\texttt{Nh}    & Non-cryptographic hash functions (Chapter \ref{ccrc0}).                     \\
\hline
\texttt{Ch}    & Cryptographic hash functions (Chapter \ref{ccrh0}).                         \\
\hline
\texttt{Sc}    & Symmetric cipher functions (Chapter \ref{ccip0}).                           \\
\hline
\texttt{Ac}    & Asymmetric cipher functions (Chapter \ref{ccip1}).                          \\
\hline
\end{tabular}
\end{center}
\end{table}

